Recall our function $L(x) = \ln x! = \ln \Gamma(x+1)$ from \eqref{LFunction}.
Although it was pivotal for our derivation of $\Gamma(x)$, we have so far neglected to explore it.
We are motivated by the fact that $L(x)$ naturally converts the product present in the expression for $\Gamma(x)$ given in \eqref{EulerForm} into a series.
Since series are some of the most troublesome characters in the solutions for many problems, any tools that $L(x)$ may provide which soothe our computational woes are deeply welcome.

Our first course of business is to take care of Euler's domain shift, while keeping the spirit of $L(x)$.
We will define the function
\begin{equation}\label{digammaAntiderivative}
\Psi^{(-1)} (x) \equiv \ln \Gamma(x).
\end{equation}
This perplexing choice of notation for $\Psi^{(-1)}$ will be justified later in this section.
For now, I ask you to simply play along.

Let's see how many properties of $\Psi^{(-1)}$ we can find before diving into any deep computations.
Firstly, our definition immediately implies that
$$\Psi^{(-1)}(x) = L(x-1).$$
From the recursion relation of $\Gamma(x)$, we know that
$$\Gamma(x+1) = x\Gamma(x),$$
so applying our definition for $\Psi^{(-1)}$ gives
\begin{equation}\label{digammaAntiderivativeRecursionFormula}
\Psi^{(-1)} (x+1) = \Psi^{(-1)} (x) + \ln x.
\end{equation}
This gives us an important recursion relation for $\Psi^{(-1)}$.
Finally, from the definition of $\Gamma(x)$, we know that it is log-convex, so $\Psi^{(-1)}$ must be a convex function.

Now let's warm up with some light computations.
Recall the remarkable Reflection Formula for $\Gamma(x)$ from \eqref{ReflectionFormula}, which says that
$$\Gamma(x)\Gamma(1-x) = \frac{\pi}{\sin (\pi x)}.$$
Taking the logarithm on both sides, we see the useful reflection formula
\begin{equation}\label{digammaAntiderivativeReflectionFormula}
\Psi^{(-1)}(x) + \Psi^{(-1)}(1-x) = \ln \pi + \ln \csc (\pi x).
\end{equation}

Recall \eqref{WeierstrassForm}, the Weierstrass form for $\Gamma(x)$, given by
$$\frac{1}{\Gamma(x)} = xe^{\gamma x} \prod_{j=1}^\infty \left( 1 + \frac{x}{j}\right) e^{-x/j}.$$
Applying the definition of $\Psi^{(-1)}(x)$,
\begin{equation}\label{digammaAntiderivativeWeierstrass}
-\Psi^{(-1)} (x) = \ln x + \gamma x + \sum_{j=1}^\infty \ln \left(1 + \frac{x}{j}\right) - \frac{x}{j}.
\end{equation}
We now expand the definition of $\gamma$, given by (4.1).
$$-\Psi^{(-1)} (x) = \ln x + \lim_{N \rightarrow \infty} -x\ln N + \sum_{j=1}^N \left(\frac{x}{j} - \frac{x}{j} + \ln \left( 1 + \frac{x}{j} \right) \right)$$
\begin{equation}\label{digammaAntiderivativeLimit}
\Psi^{(-1)} (x) = -\ln x + \lim_{N \rightarrow \infty} x\ln N - \sum_{j=1}^N \ln \left(1 + \frac{x}{j}\right)
\end{equation}
This is one expression for $\Psi^{(-1)}(x)$ which we will use in the future.

We can check this directly by recalling \eqref{EulerForm}, which gives
$$\Gamma(x) = \lim_{N \rightarrow \infty} \frac{N! N^x}{\prod_{j=0}^N (x+j)}.$$
Using the same procedure as above, we will compute $\Psi^{(-1)}(x)$.
$$\Psi^{(-1)}(x) = \lim_{N \rightarrow \infty} \ln N! + x\ln N - \sum_{j=0}^N \ln (x + j).$$
Expanding the definition of $N!$, we can simplify this expression.
$$\Psi^{(-1)}(x) = \lim_{N \rightarrow \infty} x\ln N - \ln x + \sum_{j=1}^N \ln \left( \frac{j}{x+j} \right)$$
$$\Psi^{(-1)} (x) = -\ln x + \lim_{N \rightarrow \infty} x\ln N - \sum_{j=1}^N \ln \left(1 + \frac{x}{j}\right).$$
This agrees with \eqref{digammaAntiderivativeLimit}, as expected.

So far, $\Psi^{(-1)}(x)$ has not given us anything which we could not have gotten from $\Gamma(x)$.
This is not unexpected, however, because it is simply the logarithm.
To extract new information from $\Psi^{(-1)}$, we can take the derivative.

\begin{defn}
The Digamma Function is given by
\begin{equation}\label{digammaDefinition}
\Psi(x) \equiv \frac{d \Psi^{(-1)}}{dx} = \frac{d \ln \Gamma}{dx}.
\end{equation}
\end{defn}

Now the notation $\Psi^{(-1)}(x)$ takes on the meaning of the anti-derivative of the Digamma function, justifying our previous notation.  

One may reasonably ask why we consider the \textit{logarithmic} derivative of $\Gamma(x)$, and not the regular derivative. 
For one, our expression for $\Gamma(x)$, both in the Eulerian product \eqref{EulerForm} and the Weierstrass form \eqref{WeierstrassForm}, rely on infinite products.
In general, differentiating an infinite product is messy, and does not produce many useful results in this case.
One can, of course, use the integral form \eqref{intForm} to find a simple expression for the derivative of $\Gamma(x)$, which is occasionally useful.
Nevertheless, the derivative is a linear operator, so we want to work with sums, not products.

We will now find a computable expression for \eqref{digammaDefinition}.
Starting with \eqref{digammaAntiderivativeWeierstrass}, we find that 
$$\Psi(x) = -\frac{1}{x} - \gamma + \sum_{j=1}^\infty \frac{1}{j} - \frac{\frac{1}{j}}{1 + \frac{x}{j}}$$
$$\Psi(x) = -\frac{1}{x} - \gamma + \sum_{j=1}^\infty \frac{1}{j} - \frac{1}{x + j}$$
$$\Psi(x) = -\gamma + \sum_{j=1}^\infty \frac{1}{j} - \frac{1}{x + j -1}$$
\begin{equation}\label{digammaEquation}
\boxed{\Psi(x) = -\gamma + \sum_{j=1}^\infty \frac{x - 1}{j (x + j -1)}}.
\end{equation}

Using \eqref{digammaAntiderivativeLimit} instead, we find the following.
$$\Psi(x) = -\frac{1}{x} + \lim_{N \rightarrow \infty} \ln N - \sum_{j=1}^N \frac{\frac{1}{j}}{1 + \frac{x}{j}}$$
$$\Psi(x) = -\frac{1}{x} + \lim_{N \rightarrow \infty} \ln N - \sum_{j=1}^N \frac{1}{x + j}$$
\begin{equation}\label{digammaEquationLimit}
\boxed{\Psi(x) = \lim_{N \rightarrow \infty} \ln N - \sum_{j=0}^N \frac{1}{x + j}}.
\end{equation}

Using \eqref{digammaEquation} and \eqref{digammaEquationLimit}, we can find a closed-form for many classes of sums.
We will go into more depth into this class of problems in the examples section, but for now, we can solve some especially simple cases.

First consider when $x = 1$.
From \eqref{digammaEquation}, we see that $\Psi(1) = -\gamma$.
This can also be seen from \eqref{digammaEquationLimit}, where $x = 1$ corresponds to the negative of the definition of $\gamma$.

Now consider when $x = 2$. 
From \eqref{digammaEquation}, we see that
$$\Psi(2) = -\gamma + \sum_{j=1}^\infty \frac{1}{j^2 + j}$$
$$\Psi(2) = -\gamma + \sum_{j=1}^\infty \frac{1}{j} - \frac{1}{j+1}$$
$$\Psi(2) = 1 - \gamma.$$


Now we will explore the topic of recurrence and reflection formulas for $\Psi(x)$, which have brought us great glory with $\Gamma(x)$.
From \eqref{digammaAntiderivativeRecursionFormula}, we can see that
\begin{equation}\label{digammaRecursionFormula}
\boxed{\Psi(x+1) = \Psi(x) + \frac{1}{x}},
\end{equation}
giving a simple recursion relation.
From \eqref{digammaAntiderivativeReflectionFormula}, we can see that
$$\Psi(x) - \Psi(1-x) = -\frac{d}{dx} \ln \sin (\pi x)$$
$$\Psi(x) = \Psi(1-x) + -\frac{\pi \cos (\pi x)}{\sin (\pi x)}$$
\begin{equation}\label{digammaReflectionFormula}
\boxed{\Psi(x) = \Psi(1-x) + \pi \cot (\pi x)}. 
\end{equation}

Combining the recursion formula for $\Psi(x)$, \eqref{digammaRecursionFormula}, with our observation that $\Psi(1) = -\gamma,$ we see that for $n \in \N$,
$$\Psi(n) = \sum_{j=1}^n \frac{1}{n} - \gamma,$$
showing how the Harmonic Series is connected to the Digamma function.
The appearance of $\gamma$, the definition of which required the Harmonic Series, in our computations hinted at a possible connection between our area of study and that of the Harmonic Series.
This result cements that connection, as $\Psi(x)$ is to the Harmonic Series what $\Gamma(x)$ is to the factorial function. 

Since the study of $\Psi(x)$ has given us some powerful tools, it is natural to see what other results we can squeeze out by taking further derivatives.
This motivated the definition of the \textit{Polygamma Functions}.

\begin{defn}
The Polygamma Function of order $n$ is given by 
\begin{equation}\label{polygammaDefinition}
\Psi_n (x) \equiv \frac{d^n}{dx^n} \Psi(x) = \frac{d^{n+1}}{dx^{n+1}} (\ln \Gamma(x)),
\end{equation}
where $\Psi_0 (x) \equiv \Psi(x)$.
\end{defn}

It is not so tedious to derive a closed form for \eqref{polygammaDefinition} if we start from \eqref{digammaEquationLimit}.
If we start with \eqref{digammaEquation} instead, we need to handle the quotient of products in the infinite series, creating a tractable but ultimately unnecessary mess.
$$\Psi_n(x) = \frac{d^n}{dx^n} \Psi(x)$$
$$\Psi_n(x) = \frac{d^n}{dx^n} \left[ \lim_{N \rightarrow \infty} \ln N - \sum_{j=0}^N \frac{1}{x+j} \right]$$
$$\Psi_n(x) = -\frac{d^n}{dx^n} \sum_{j=0}^\infty \frac{1}{x+j} \hspace{10px} (\text{Assuming $n \geq 1$})$$
\begin{equation}\label{polygammaEquation}
\boxed{\Psi_n(x) = \sum_{j=0}^\infty \frac{(-1)^{n+1} n!}{(x+j)^{n+1}}} \hspace{10px} \text{for $n \geq 1$}.
\end{equation}

We will finish this section by using \eqref{digammaRecursionFormula} to find a recursion formula for $\Psi_n (x)$.
We can see that
\begin{equation}\label{polygammaRecursionFormula}
\boxed{\Psi_n (x+1) = \Psi_n (x) + \frac{(-1)^{n} n!}{x^{n+1}}}.
\end{equation}
Notice the striking similarity to \eqref{polygammaEquation}, due to the artifact of repeated differentiation of a power, which we had originally used to find the integral form for $\Gamma(x)$.