\begin{defn}
Recall that the factorial function, written $n!$, is a map on $\mathbb{N}$ given by the formula
\begin{equation}\label{factorialDef}
    n! \equiv n \cdot (n-1) \cdot (n-2) \cdots 1.
\end{equation}
\end{defn}
The $\Gamma$ function, in short, is the ``natural" extension of the factorial function beyond the natural numbers.
As such, we will enumerate the conditions of the standard factorial function which we want to preserve.
The first of these conditions is the following recursion relation.

\begin{pro}
The factorial function obeys the recursion relation
\begin{equation}\label{factorialRecursion}
n! = n \cdot (n-1)!.
\end{equation}
\end{pro}

To see this, consider \eqref{factorialDef}.
All terms from $n-1$ to $1$ are themselves $(n-1)!$ by definition.
Using \eqref{factorialRecursion}, given some $n$, we can find $(n+1)!$ and $(n-1)!$ by the following formulas.

\begin{equation}\label{factorialRecursionForward}
(n+1)! = (n+1)n!
\end{equation}
\begin{equation}\label{factorialRecursionBackward}
(n-1)! = \frac{n!}{n}
\end{equation}

By \eqref{factorialDef}, we see that $1! = 1$. 
Applying \eqref{factorialRecursionBackward} to this, we see that $0! = 1$.
If we then try to compute $-1!$ by the same technique, we find that it is undefined because of a division by zero.
In fact, this will happen for all negative integers.
Therefore, whatever the form of $\Gamma$ turns out to be, it must be undefined on the negative integers.

By repeated application of \eqref{factorialRecursion}, we can find a more general recursion relation given by 
\begin{equation}\label{factorialSuperRecursion}
(n + k)! = n! \prod_{j=1}^k (n + j).
\end{equation}
This relation will be crucial for our derivation.

\eqref{factorialRecursionForward} and \eqref{factorialRecursionBackward} simplify our problem of generalizing the factorials significantly. 
\begin{obs}
Notice that once $x!$ is determined on \textit{any} closed interval of unit length, it is determined over all of $\mathbb{R}$.
\end{obs}

\begin{proof}
Take any $x \in \mathbb{R}$. 
Suppose $x!$ is defined on the unit interval $I = [a,a + 1]$ for some $a \in \mathbb{R}$.
If $x \in I$, then we are done.
Otherwise, we know by properties of the integers that we can find some $n \in \mathbb{Z}$ such that $x + n$ is in $I$. 
Then, by repeated application of \eqref{factorialRecursionForward} or \eqref{factorialRecursionBackward} on $x + n$ will give us $x!$. 
\end{proof}

Observation (2.8) gives us the key to our derivation - we need only find one ``small" part of $\mathbb{R}$ where $n!$ is easy to extend.
Unfortunately, it is not so easy to find such an interval for $n!$ itself.
This difficulty stems for the rapid growth of $n!,$ which we may see by approximating its derivative.
$$\frac{d}{dn} (n!) \approx n! - (n-1)!$$
$$\frac{d}{dn} (n!) \approx (n-1) (n-1)!$$

From this derivative approximation, we can see that the derivative of the factorial is a factorial, similar to exponentials.
This indicates that factorials grow at least as fast as exponentials. 
As an exercise, one may show factorials grow even faster that exponentials (hint: both exponentials and factorials are repeated multiplications, but factorials repeatedly multiply by terms that grow larger linearly, while exponentials repeatedly multiply by constant terms).

To tame this growth, we will consider the following function.
Let $L(n) = \ln (n!)$. 
Applying logarithm rules to \eqref{factorialDef}, we find the formula
\begin{equation}\label{LFunction}
L(n) = \sum_{j=1}^n \ln j.
\end{equation}

Additionally, from \eqref{factorialSuperRecursion}, we see the generalized recursion relation 
\begin{equation}\label{LFunctionRecursion}
L(N + n) = L(N) + \sum_{j=1}^n \ln(N+j).
\end{equation}

Now, consider $\ln (N + k) - \ln N$ for $N \gg k$. 
We know that
$$\ln (N + k) - \ln N = \ln\left(1 + \frac{k}{N}\right).$$
Taking the limit formally as $N \rightarrow \infty$, we see that $\ln (N+k) - \ln N \rightarrow 0$.
We've shown roughly that $$\ln (N+k) \approx \ln N$$.

Revisiting \eqref{LFunctionRecursion} With this new approximation, we can say that
\begin{equation}\label{LFunctionApproxRecursion}
L(N+n) \approx L(N) + n \ln N.
\end{equation}
Notice that \eqref{LFunctionApproxRecursion} no longer requires $n$ to be an integer, so we have effectively defined $L(x)$ over an interval using the fact that it looks almost linear for large enough $N$.
Of course, \eqref{LFunctionApproxRecursion} cannot be ``proven" in any sense.
We are asserting that this property should hold for $L(x)$ in the limit of large $N$.
We can symbolically show this definition over the interval by changing variables from $n$ to $x$.

Continuing from \eqref{LFunctionApproxRecursion}, we can use \eqref{LFunction} and \eqref{LFunctionRecursion} to solve for $L(x)$.
$$L(x) + \sum_{j = 1}^N \ln (x+j) \approx \sum_{j=1}^N \ln j + x \ln N$$
$$L(x) \approx x\ln N + \sum_{j=1}^N \ln \left( \frac{j}{x+j} \right)$$
Taking the limit as $N \rightarrow \infty$ formally, we find an exact equality.
$$L(x) = \lim_{N \rightarrow \infty} x \ln N + \sum_{j=1}^N \ln \left( \frac{j}{j+x} \right)$$
Exponentiating, we find our desired generalized factorial.
$$\boxed{x! = \lim_{N \rightarrow \infty} N^x \prod_{j=1}^N \frac{j}{x+j}}$$
Writing this another way which makes clear the connection to the standard factorial, we get
$$x! = \lim_{N \rightarrow \infty} \frac{N! N^x}{\prod_{j=1}^N (x+j)}.$$

Notice that this formula has poles at the negative integers, because the product is undefined when $j = -x$.
This formula agrees with our expected behaviour in the negative integer case.
We now show that it exists for all other real numbers.

\begin{prop}
The limit in the equation 
$$x! = \lim_{N \rightarrow \infty} N^x \prod_{j=1}^N \frac{j}{x+j}$$
exists for all real numbers $x$ except negative integers.
\end{prop}

\begin{proof}
Take an arbitrary real number $x$ which is not a negative integer.
Note that we can rewrite the generalized factorial as 
$$x! = \lim_{N \rightarrow \infty} \left[e^{-x \ln N} \prod_{j=1}^N \left(1 + \frac{x}{j}\right) \right]^{-1}.$$
We now multiply by a strategic factor of $e^{-x/j}$ for each $j$ in the product.
\begin{equation}\label{WeierstrassFactorial}
    x! = \lim_{N \rightarrow \infty} \left[ e^{x(\sum_{j=1}^N (1/j) - \ln N)} \prod_{j=1}^N \left(1 + \frac{x}{j}\right) e^{-x/j} \right]^{-1}.
\end{equation}
\eqref{WeierstrassFactorial} is in fact another way to define the $\Gamma$ function, and was historically employed by Weierstrass. 
Since $e^x$ is continuous, it is sufficient to show that the following two sums converge.
Consider
$$S_1 \equiv \lim_{N \rightarrow \infty} \left( \sum_{j=1}^N \left(\frac{1}{j}\right) - \ln N \right).$$
By the integral test, one can show that $S_1$ converges.
In fact, it converges to a special constant known as the \textit{Euler-Mascheroni constant}, written $\gamma$.
Now consider the sum
$$S_2 \equiv \sum_{j=1}^\infty \ln \left( 1 + \frac{x}{j} \right) - \frac{x}{j}.$$
This sum converges if and only if the product
$$\prod_{j=1}^\infty \left(1 + \frac{x}{j}\right) e^{-x/j}$$
converges.
We can see that $S_2$ converges by again applying the integral test.
Therefore, $x!$ exists for our desired domain.
\end{proof}

Having done a ``derivation" of the $\Gamma$ function, we are now ready to formalize our findings.

\begin{defn}\label{BohrMollerupDef}
$\Gamma: \mathbb{R} - \mathbb{Z}_{< 0} \rightarrow \mathbb{R}$ is a function satisfying the following three conditions.
\begin{enumerate}
    \item for any natural number $n$, $\Gamma (n+1) = n!$
    \item for any real number $x$, $\Gamma (x+1) = x \Gamma(x)$
    \item $\Gamma(x)$ is logarithmically convex on $(0, \infty)$
\end{enumerate}
\end{defn}

At this point, there are many questions about this definition.
We will attempt to elucidate the answers to some of them, but we will inevitably ignore some others.
As such, we recommend the reader to think deeply about this definition.

The first question is regarding the domain shift by one in the definition of $\Gamma(n+1) = n!$.
This is due to historical reasons (for this peculiarity, as with many others, we have Euler to blame).

The second question is if our formula for $\Gamma(x)$ satisfies the conditions. 
We claim that the formula
\begin{equation}\label{EulerForm}
\boxed{\Gamma(x) \equiv \lim_{N \rightarrow \infty} \frac{N! N^x}{\prod_{j=0}^N (x+j)}}
\end{equation}
does satisfy this definition.
We can see that condition two is the recursion relation which we took as central at the start of our derivation.
Similarly, condition three corresponds to our loose argument that $x!$ can be linearly interpolated accurately for large $x$.
We will show this formally in the next theorem.

\begin{thm}
\eqref{EulerForm} satisfies the definition of the $\Gamma$ function.
\end{thm}

\begin{proof}
We start by showing condition one.
Take $n \in \N$.
By our equation, we see
$$\Gamma(n+1) = \lim_{N \rightarrow \infty} \frac{N!N^{n+1}}{\prod_{j=0}^N (n+1+j)}$$
$$\Gamma(n+1) = \lim_{N \rightarrow \infty} \frac{N!N^{n+1}}{\prod_{j=1}^{N+1} (n+j)}$$
Then notice that 
$$\prod_{j=1}^{N+1} (n+j) = \frac{(n+N+1)!}{n!},$$
so we get 
$$\Gamma(n+1) = \lim_{N \rightarrow \infty} \frac{n! N! N^{n+1}}{(n+N+1)!}$$
$$\Gamma(n+1) = n! \lim_{N \rightarrow \infty} \frac{N! N^{n+1}}{(n+N+1)!}.$$
Now let 
$$\mathcal{L}(n) = \lim_{N \rightarrow \infty} \frac{N! N^n}{(n+N)!}.$$
We want to show that $\mathcal{L} = 1$, as this is equivalent to the limit in the expression for $\Gamma(n+1)$ if we send $n \mapsto n+1$.
Cancelling terms, we see that
$$\mathcal{L}(n) = \lim_{N \rightarrow \infty} \frac{N^n}{(N+1)\cdots(N+n)}$$
As $N \rightarrow \infty$, $N + n \approx N$, so the denominator becomes $N^n$.
This shows that $\mathcal{L}(n) = 1$.
Therefore, $\forall n \in \N, \Gamma(n+1) = n!$.

Now consider condition two. 
Take any $x \in \R - \Z_{< 0}$.
Applying our formula, 
$$\Gamma(x+1) = \lim_{N \rightarrow \infty} \frac{N!N^{x+1}}{\prod_{j=0}^N (x + j + 1)}$$
$$\Gamma(x+1) = \lim_{N \rightarrow \infty} \frac{Nx}{(x+N+1)} \frac{N! N^x}{\prod_{j=0}^N (x+j)}$$
$$\Gamma(x+1) = \lim_{N \rightarrow \infty} \frac{Nx}{x+N+1} \lim_{N \rightarrow \infty} \frac{N! N^x}{\prod_{j=0}^N (x+j)}$$
$$\Gamma(x+1) = x \Gamma(x),$$
so our formula satisfies condition two. 

Finally, we show that $\Gamma(x)$ is logarithmically convex. 
First, we compute that 
$$\ln \Gamma (x) = \lim_{N \rightarrow \infty} x\ln N - \ln x + \sum_{j=1}^N \left(\ln\frac{j}{x+j}\right).$$
Now consider arbitrary $a, b \in (0, \infty)$ and $t \in [0,1]$.
Without loss of generality, assume that $b > a$.
We want to show that
$$\ln\Gamma(ta + (1-t)b) \leq t\ln\Gamma(a) + (1-t)\ln\Gamma(b).$$
This is equivalent to showing
\begin{multline*}
    \lim_{N \rightarrow \infty} (ta + (1-t)b)\ln N - \ln (ta + (1-t)b) + \sum_{j=1}^N \left( \ln \frac{j}{ta + (1-t)b + j} \right) \leq \\ \lim_{N \rightarrow \infty} 
    (ta + (1-t)b)\ln N - t\ln a - (1-t)\ln b + \sum_{j=1}^N \left( t\ln \frac{j}{a+j} + (1-t) \ln \frac{j}{b+j} \right).
\end{multline*}
We see that the first term cancels.
Additionally, since $\ln x$ is concave, $-\ln x$ is convex, so the inequality holds for the second and third terms. 
By the same argument, and some log properties, one can show that the sums also obey the inequality.
Therefore, $\Gamma(x)$ is logarithmically convex, and our equation satisfies our definition. 
\end{proof}

Now that we have a formula for $\Gamma(x)$, it would be nice to show that it is the only formula.
Uniqueness allows us to assert that if a function satisfies the definition of the $\Gamma$ function, then it is equivalent to all of our previous representations, hence saving us the work of reproving all its properties.
On the other hand, the proof of this theorem is long, formal, and largely a repeat of our derivation, so we shall not prove it here.

\begin{thm}
(Bohr-Mollerup Theorem) Definition \ref{BohrMollerupDef} specifies a unique function.
\end{thm}

\begin{proof}
Two proofs are given by B\"{o}rgers and Duong \cite{uniquenessPF}\cite{uniquenessPF2}.
\end{proof}

