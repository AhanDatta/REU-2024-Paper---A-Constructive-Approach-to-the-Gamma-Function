\documentclass[openany, amssymb, psamsfonts]{amsart}
\usepackage{mathrsfs,comment}
\usepackage[usenames,dvipsnames]{color}
\usepackage[normalem]{ulem}
\usepackage{nicematrix}
\usepackage{booktabs}
\usepackage{tikz}
\usepackage{url}
\usepackage[all,arc,2cell]{xy}
\UseAllTwocells
\usepackage{enumerate}
%%% hyperref stuff is taken from AGT style file
\usepackage{hyperref}  
\hypersetup{%
  bookmarksnumbered=true,%
  bookmarks=true,%
  colorlinks=true,%
  linkcolor=blue,%
  citecolor=blue,%
  filecolor=blue,%
  menucolor=blue,%
  pagecolor=blue,%
  urlcolor=blue,%
  pdfnewwindow=true,%
  pdfstartview=FitBH}   
  
\let\fullref\autoref
%
%  \autoref is very crude.  It uses counters to distinguish environments
%  so that if say {lemma} uses the {theorem} counter, then autrorefs
%  which should come out Lemma X.Y in fact come out Theorem X.Y.  To
%  correct this give each its own counter eg:
%                 \newtheorem{theorem}{Theorem}[section]
%                 \newtheorem{lemma}{Lemma}[section]
%  and then equate the counters by commands like:
%                 \makeatletter
%                   \let\c@lemma\c@theorem
%                  \makeatother
%
%  To work correctly the environment name must have a corrresponding 
%  \XXXautorefname defined.  The following command does the job:
%
\def\makeautorefname#1#2{\expandafter\def\csname#1autorefname\endcsname{#2}}
%
%  Some standard autorefnames.  If the environment name for an autoref 
%  you need is not listed below, add a similar line to your TeX file:
%  
%\makeautorefname{equation}{Equation}%
\def\equationautorefname~#1\null{(#1)\null}
\makeautorefname{footnote}{footnote}%
\makeautorefname{item}{item}%
\makeautorefname{figure}{Figure}%
\makeautorefname{table}{Table}%
\makeautorefname{part}{Part}%
\makeautorefname{appendix}{Appendix}%
\makeautorefname{chapter}{Chapter}%
\makeautorefname{section}{Section}%
\makeautorefname{subsection}{Section}%
\makeautorefname{subsubsection}{Section}%
\makeautorefname{theorem}{Theorem}%
\makeautorefname{thm}{Theorem}%
\makeautorefname{cor}{Corollary}%
\makeautorefname{lem}{Lemma}%
\makeautorefname{prop}{Proposition}%
\makeautorefname{pro}{Property}
\makeautorefname{conj}{Conjecture}%
\makeautorefname{defn}{Definition}%
\makeautorefname{notn}{Notation}
\makeautorefname{notns}{Notations}
\makeautorefname{rem}{Remark}%
\makeautorefname{quest}{Question}%
\makeautorefname{exmp}{Example}%
\makeautorefname{ax}{Axiom}%
\makeautorefname{claim}{Claim}%
\makeautorefname{ass}{Assumption}%
\makeautorefname{asss}{Assumptions}%
\makeautorefname{con}{Construction}%
\makeautorefname{prob}{Problem}%
\makeautorefname{warn}{Warning}%
\makeautorefname{obs}{Observation}%
\makeautorefname{conv}{Convention}%


%
%                  *** End of hyperref stuff ***

%Convinience commands
\newcommand{\N}{\mathbb{N}}
\newcommand{\Z}{\mathbb{Z}}
\newcommand{\R}{\mathbb{R}}
\newcommand{\cd}{\cdot}

%theoremstyle{plain} --- default
\newtheorem{thm}{Theorem}[section]
\newtheorem{cor}{Corollary}[section]
\newtheorem{prop}{Proposition}[section]
\newtheorem{lem}{Lemma}[section]
\newtheorem{prob}{Problem}[section]
\newtheorem{conj}{Conjecture}[section]
%\newtheorem{ass}{Assumption}[section]
%\newtheorem{asses}{Assumptions}[section]

\theoremstyle{definition}
\newtheorem{defn}{Definition}[section]
\newtheorem{ass}{Assumption}[section]
\newtheorem{asss}{Assumptions}[section]
\newtheorem{ax}{Axiom}[section]
\newtheorem{con}{Construction}[section]
\newtheorem{exmp}{Example}[section]
\newtheorem{notn}{Notation}[section]
\newtheorem{notns}{Notations}[section]
\newtheorem{pro}{Property}[section]
\newtheorem{quest}{Question}[section]
\newtheorem{rem}{Remark}[section]
\newtheorem{warn}{Warning}[section]
\newtheorem{sch}{Scholium}[section]
\newtheorem{obs}{Observation}[section]
\newtheorem{conv}{Convention}[section]

%%%% hack to get fullref working correctly
\makeatletter
\let\c@obs=\c@thm
\let\c@cor=\c@thm
\let\c@prop=\c@thm
\let\c@lem=\c@thm
\let\c@prob=\c@thm
\let\c@con=\c@thm
\let\c@conj=\c@thm
\let\c@defn=\c@thm
\let\c@notn=\c@thm
\let\c@notns=\c@thm
\let\c@exmp=\c@thm
\let\c@ax=\c@thm
\let\c@pro=\c@thm
\let\c@ass=\c@thm
\let\c@warn=\c@thm
\let\c@rem=\c@thm
\let\c@sch=\c@thm
\let\c@equation\c@thm
\numberwithin{equation}{section}
\makeatother

\bibliographystyle{plain}

%--------Meta Data: Fill in your info------
\title{The Gamma Function: A Constructive Approach}

\author{Ahan Datta}

\begin{document}

\begin{abstract}

This paper motivates the study of the $\Gamma$ function for a practical audience.
For accessibility to non-mathematicians, we limit our analysis to $\R$.
We derive the most important formulas and properties of the $\Gamma$ function for the practical student with clear, easy-to-follow arguments.
Finally, we show the power of the $\Gamma$ function by computing some example integrals and sums.

\end{abstract}

\maketitle

\tableofcontents

\section{Introduction}

    The $\Gamma$ function is useful in the solution developments for many physics problems.
When one first encounters it as a physicist, however, it is often presented far too opaquely. 
In this paper, however, you will find first-principles derivations of the most important properties and formulas for the $\Gamma$ function, in a manner familiar to the aspiring physicist.

We do assume some previous knowledge of analysis in $\R$, such as convexity.
Nevertheless, as a general rule, we have attempted to limit the methods in this paper to those most common in physics.
Additionally, we emphasize the $\Gamma$ function's use in solving real integrals and sums.
As such, given that you have some mental fortitude, this paper will give you the necessary tools to understand the $\Gamma$ function's role in mathematical physics.

\newpage
\section{Deriving the Gamma Function}

    \begin{defn}
Recall that the factorial function, written $n!$, is a map on $\mathbb{N}$ given by the formula
\begin{equation}\label{factorialDef}
    n! \equiv n \cdot (n-1) \cdot (n-2) \cdots 1.
\end{equation}
\end{defn}
The $\Gamma$ function, in short, is the ``natural" extension of the factorial function beyond the natural numbers.
As such, we will enumerate the conditions of the standard factorial function which we want to preserve.
The first of these conditions is the following recursion relation.

\begin{pro}
The factorial function obeys the recursion relation
\begin{equation}\label{factorialRecursion}
n! = n \cdot (n-1)!.
\end{equation}
\end{pro}

To see this, consider \eqref{factorialDef}.
All terms from $n-1$ to $1$ are themselves $(n-1)!$ by definition.
Using \eqref{factorialRecursion}, given some $n$, we can find $(n+1)!$ and $(n-1)!$ by the following formulas.

\begin{equation}\label{factorialRecursionForward}
(n+1)! = (n+1)n!
\end{equation}
\begin{equation}\label{factorialRecursionBackward}
(n-1)! = \frac{n!}{n}
\end{equation}

By \eqref{factorialDef}, we see that $1! = 1$. 
Applying \eqref{factorialRecursionBackward} to this, we see that $0! = 1$.
If we then try to compute $-1!$ by the same technique, we find that it is undefined because of a division by zero.
In fact, this will happen for all negative integers.
Therefore, whatever the form of $\Gamma$ turns out to be, it must be undefined on the negative integers.

By repeated application of \eqref{factorialRecursion}, we can find a more general recursion relation given by 
\begin{equation}\label{factorialSuperRecursion}
(n + k)! = n! \prod_{j=1}^k (n + j).
\end{equation}
This relation will be crucial for our derivation.

\eqref{factorialRecursionForward} and \eqref{factorialRecursionBackward} simplify our problem of generalizing the factorials significantly. 
\begin{obs}
Notice that once $x!$ is determined on \textit{any} closed interval of unit length, it is determined over all of $\mathbb{R}$.
\end{obs}

\begin{proof}
Take any $x \in \mathbb{R}$. 
Suppose $x!$ is defined on the unit interval $I = [a,a + 1]$ for some $a \in \mathbb{R}$.
If $x \in I$, then we are done.
Otherwise, we know by properties of the integers that we can find some $n \in \mathbb{Z}$ such that $x + n$ is in $I$. 
Then, by repeated application of \eqref{factorialRecursionForward} or \eqref{factorialRecursionBackward} on $x + n$ will give us $x!$. 
\end{proof}

Observation (2.8) gives us the key to our derivation - we need only find one ``small" part of $\mathbb{R}$ where $n!$ is easy to extend.
Unfortunately, it is not so easy to find such an interval for $n!$ itself.
This difficulty stems for the rapid growth of $n!,$ which we may see by approximating its derivative.
$$\frac{d}{dn} (n!) \approx n! - (n-1)!$$
$$\frac{d}{dn} (n!) \approx (n-1) (n-1)!$$

From this derivative approximation, we can see that the derivative of the factorial is a factorial, similar to exponentials.
This indicates that factorials grow at least as fast as exponentials. 
As an exercise, one may show factorials grow even faster that exponentials (hint: both exponentials and factorials are repeated multiplications, but factorials repeatedly multiply by terms that grow larger linearly, while exponentials repeatedly multiply by constant terms).

To tame this growth, we will consider the following function.
Let $L(n) = \ln (n!)$. 
Applying logarithm rules to \eqref{factorialDef}, we find the formula
\begin{equation}\label{LFunction}
L(n) = \sum_{j=1}^n \ln j.
\end{equation}

Additionally, from \eqref{factorialSuperRecursion}, we see the generalized recursion relation 
\begin{equation}\label{LFunctionRecursion}
L(N + n) = L(N) + \sum_{j=1}^n \ln(N+j).
\end{equation}

Now, consider $\ln (N + k) - \ln N$ for $N \gg k$. 
We know that
$$\ln (N + k) - \ln N = \ln\left(1 + \frac{k}{N}\right).$$
Taking the limit formally as $N \rightarrow \infty$, we see that $\ln (N+k) - \ln N \rightarrow 0$.
We've shown roughly that $$\ln (N+k) \approx \ln N$$.

Revisiting \eqref{LFunctionRecursion} With this new approximation, we can say that
\begin{equation}\label{LFunctionApproxRecursion}
L(N+n) \approx L(N) + n \ln N.
\end{equation}
Notice that \eqref{LFunctionApproxRecursion} no longer requires $n$ to be an integer, so we have effectively defined $L(x)$ over an interval using the fact that it looks almost linear for large enough $N$.
Of course, \eqref{LFunctionApproxRecursion} cannot be ``proven" in any sense.
We are asserting that this property should hold for $L(x)$ in the limit of large $N$.
We can symbolically show this definition over the interval by changing variables from $n$ to $x$.

Continuing from \eqref{LFunctionApproxRecursion}, we can use \eqref{LFunction} and \eqref{LFunctionRecursion} to solve for $L(x)$.
$$L(x) + \sum_{j = 1}^N \ln (x+j) \approx \sum_{j=1}^N \ln j + x \ln N$$
$$L(x) \approx x\ln N + \sum_{j=1}^N \ln \left( \frac{j}{x+j} \right)$$
Taking the limit as $N \rightarrow \infty$ formally, we find an exact equality.
$$L(x) = \lim_{N \rightarrow \infty} x \ln N + \sum_{j=1}^N \ln \left( \frac{j}{j+x} \right)$$
Exponentiating, we find our desired generalized factorial.
$$\boxed{x! = \lim_{N \rightarrow \infty} N^x \prod_{j=1}^N \frac{j}{x+j}}$$
Writing this another way which makes clear the connection to the standard factorial, we get
$$x! = \lim_{N \rightarrow \infty} \frac{N! N^x}{\prod_{j=1}^N (x+j)}.$$

Notice that this formula has poles at the negative integers, because the product is undefined when $j = -x$.
This formula agrees with our expected behaviour in the negative integer case.
We now show that it exists for all other real numbers.

\begin{prop}
The limit in the equation 
$$x! = \lim_{N \rightarrow \infty} N^x \prod_{j=1}^N \frac{j}{x+j}$$
exists for all real numbers $x$ except negative integers.
\end{prop}

\begin{proof}
Take an arbitrary real number $x$ which is not a negative integer.
Note that we can rewrite the generalized factorial as 
$$x! = \lim_{N \rightarrow \infty} \left[e^{-x \ln N} \prod_{j=1}^N \left(1 + \frac{x}{j}\right) \right]^{-1}.$$
We now multiply by a strategic factor of $e^{-x/j}$ for each $j$ in the product.
\begin{equation}\label{WeierstrassFactorial}
    x! = \lim_{N \rightarrow \infty} \left[ e^{x(\sum_{j=1}^N (1/j) - \ln N)} \prod_{j=1}^N \left(1 + \frac{x}{j}\right) e^{-x/j} \right]^{-1}.
\end{equation}
\eqref{WeierstrassFactorial} is in fact another way to define the $\Gamma$ function, and was historically employed by Weierstrass. 
Since $e^x$ is continuous, it is sufficient to show that the following two sums converge.
Consider
$$S_1 \equiv \lim_{N \rightarrow \infty} \left( \sum_{j=1}^N \left(\frac{1}{j}\right) - \ln N \right).$$
By the integral test, one can show that $S_1$ converges.
In fact, it converges to a special constant known as the \textit{Euler-Mascheroni constant}, written $\gamma$.
Now consider the sum
$$S_2 \equiv \sum_{j=1}^\infty \ln \left( 1 + \frac{x}{j} \right) - \frac{x}{j}.$$
This sum converges if and only if the product
$$\prod_{j=1}^\infty \left(1 + \frac{x}{j}\right) e^{-x/j}$$
converges.
We can see that $S_2$ converges by again applying the integral test.
Therefore, $x!$ exists for our desired domain.
\end{proof}

Having done a ``derivation" of the $\Gamma$ function, we are now ready to formalize our findings.

\begin{defn}\label{BohrMollerupDef}
$\Gamma: \mathbb{R} - \mathbb{Z}_{< 0} \rightarrow \mathbb{R}$ is a function satisfying the following three conditions.
\begin{enumerate}
    \item for any natural number $n$, $\Gamma (n+1) = n!$
    \item for any real number $x$, $\Gamma (x+1) = x \Gamma(x)$
    \item $\Gamma(x)$ is logarithmically convex on $(0, \infty)$
\end{enumerate}
\end{defn}

At this point, there are many questions about this definition.
We will attempt to elucidate the answers to some of them, but we will inevitably ignore some others.
As such, we recommend the reader to think deeply about this definition.

The first question is regarding the domain shift by one in the definition of $\Gamma(n+1) = n!$.
This is due to historical reasons (for this peculiarity, as with many others, we have Euler to blame).

The second question is if our formula for $\Gamma(x)$ satisfies the conditions. 
We claim that the formula
\begin{equation}\label{EulerForm}
\boxed{\Gamma(x) \equiv \lim_{N \rightarrow \infty} \frac{N! N^x}{\prod_{j=0}^N (x+j)}}
\end{equation}
does satisfy this definition.
We can see that condition two is the recursion relation which we took as central at the start of our derivation.
Similarly, condition three corresponds to our loose argument that $x!$ can be linearly interpolated accurately for large $x$.
We will show this formally in the next theorem.

\begin{thm}
\eqref{EulerForm} satisfies the definition of the $\Gamma$ function.
\end{thm}

\begin{proof}
We start by showing condition one.
Take $n \in \N$.
By our equation, we see
$$\Gamma(n+1) = \lim_{N \rightarrow \infty} \frac{N!N^{n+1}}{\prod_{j=0}^N (n+1+j)}$$
$$\Gamma(n+1) = \lim_{N \rightarrow \infty} \frac{N!N^{n+1}}{\prod_{j=1}^{N+1} (n+j)}$$
Then notice that 
$$\prod_{j=1}^{N+1} (n+j) = \frac{(n+N+1)!}{n!},$$
so we get 
$$\Gamma(n+1) = \lim_{N \rightarrow \infty} \frac{n! N! N^{n+1}}{(n+N+1)!}$$
$$\Gamma(n+1) = n! \lim_{N \rightarrow \infty} \frac{N! N^{n+1}}{(n+N+1)!}.$$
Now let 
$$\mathcal{L}(n) = \lim_{N \rightarrow \infty} \frac{N! N^n}{(n+N)!}.$$
We want to show that $\mathcal{L} = 1$, as this is equivalent to the limit in the expression for $\Gamma(n+1)$ if we send $n \mapsto n+1$.
Cancelling terms, we see that
$$\mathcal{L}(n) = \lim_{N \rightarrow \infty} \frac{N^n}{(N+1)\cdots(N+n)}$$
As $N \rightarrow \infty$, $N + n \approx N$, so the denominator becomes $N^n$.
This shows that $\mathcal{L}(n) = 1$.
Therefore, $\forall n \in \N, \Gamma(n+1) = n!$.

Now consider condition two. 
Take any $x \in \R - \Z_{< 0}$.
Applying our formula, 
$$\Gamma(x+1) = \lim_{N \rightarrow \infty} \frac{N!N^{x+1}}{\prod_{j=0}^N (x + j + 1)}$$
$$\Gamma(x+1) = \lim_{N \rightarrow \infty} \frac{Nx}{(x+N+1)} \frac{N! N^x}{\prod_{j=0}^N (x+j)}$$
$$\Gamma(x+1) = \lim_{N \rightarrow \infty} \frac{Nx}{x+N+1} \lim_{N \rightarrow \infty} \frac{N! N^x}{\prod_{j=0}^N (x+j)}$$
$$\Gamma(x+1) = x \Gamma(x),$$
so our formula satisfies condition two. 

Finally, we show that $\Gamma(x)$ is logarithmically convex. 
First, we compute that 
$$\ln \Gamma (x) = \lim_{N \rightarrow \infty} x\ln N - \ln x + \sum_{j=1}^N \left(\ln\frac{j}{x+j}\right).$$
Now consider arbitrary $a, b \in (0, \infty)$ and $t \in [0,1]$.
Without loss of generality, assume that $b > a$.
We want to show that
$$\ln\Gamma(ta + (1-t)b) \leq t\ln\Gamma(a) + (1-t)\ln\Gamma(b).$$
This is equivalent to showing
\begin{multline*}
    \lim_{N \rightarrow \infty} (ta + (1-t)b)\ln N - \ln (ta + (1-t)b) + \sum_{j=1}^N \left( \ln \frac{j}{ta + (1-t)b + j} \right) \leq \\ \lim_{N \rightarrow \infty} 
    (ta + (1-t)b)\ln N - t\ln a - (1-t)\ln b + \sum_{j=1}^N \left( t\ln \frac{j}{a+j} + (1-t) \ln \frac{j}{b+j} \right).
\end{multline*}
We see that the first term cancels.
Additionally, since $\ln x$ is concave, $-\ln x$ is convex, so the inequality holds for the second and third terms. 
By the same argument, and some log properties, one can show that the sums also obey the inequality.
Therefore, $\Gamma(x)$ is logarithmically convex, and our equation satisfies our definition. 
\end{proof}

Now that we have a formula for $\Gamma(x)$, it would be nice to show that it is the only formula.
Uniqueness allows us to assert that if a function satisfies the definition of the $\Gamma$ function, then it is equivalent to all of our previous representations, hence saving us the work of reproving all its properties.
On the other hand, the proof of this theorem is long, formal, and largely a repeat of our derivation, so we shall not prove it here.

\begin{thm}
(Bohr-Mollerup Theorem) Definition \ref{BohrMollerupDef} specifies a unique function.
\end{thm}

\begin{proof}
Two proofs are given by B\"{o}rgers and Duong \cite{uniquenessPF}\cite{uniquenessPF2}.
\end{proof}



\newpage
\section{Alternative Representations}

    We will start with by using our newfound knowledge to compute $(1/2)! = \Gamma(3/2)$.
First, however, we need to make a slight modification to our formula to recover Euler's historically first expression for $\Gamma(x)$.

\begin{prop}
\begin{equation}\label{historicProductForm}
\Gamma(x+1) = \prod_{j=1}^\infty \left( \frac{\left(1 + \frac{1}{j}\right)^x}{1 + \frac{x}{j}}\ \right).
\end{equation}
\end{prop}

\begin{proof}
First recall that
$$\Gamma(x+1) = \lim_{N \rightarrow \infty} N^x \prod_{j=1}^N \left( \frac{j}{j+x} \right).$$
Dividing through by $j$ inside the product, we find
$$\Gamma(x+1) = \lim_{N \rightarrow \infty} N^x \prod_{j=1}^N \frac{1}{1 + \frac{x}{j}}.$$
All that is left to show now is that, in the limit,
$$N^x \approx \prod_{j=1}^N \left(1 + \frac{1}{j}\right)^x.$$
This is true if
$$\prod_{j=1}^N \left(1 + \frac{1}{j}\right) = N+1,$$
which we will show by induction on $N$.
The base case of $N=1$ is clear.
For the inductive step, take some $k \in \N$ such that 
$$\prod_{j=1}^k \left(1 + \frac{1}{j}\right) = k+1.$$
Then, the product up to $k+1$ can be manipulated as follows.
\begin{align*}
\prod_{j=1}^{k+1} \left(1 + \frac{1}{j}\right) &= \left(1+\frac{1}{k+1}\right)\prod_{j=1}^{k} \left(1 + \frac{1}{j}\right) \\
&= \left(1+\frac{1}{k+1}\right) (k+1) \\
&= (k+1) + 1.
\end{align*}
Therefore, our inductive step is fulfilled, our desired limit is true, and
$$\Gamma(x+1) = \prod_{j=1}^\infty \left( \frac{\left(1 + \frac{1}{j}\right)^{x}}{1 + \frac{x}{j}} \right).$$
\end{proof}

Now, computing $(1/2)!$ is a matter of substituting $x=1/2$ into \eqref{historicProductForm}.
\begin{align*}
\Gamma(3/2) &= \prod_{j=1}^\infty \frac{\left(1 + \frac{1}{j}\right)^{1/2}}{1 + \frac{1}{2j}} \\
&= \left(\prod_{j=1}^\infty \frac{1 + \frac{1}{j}}{\left(1 + \frac{1}{2j}\right)^2}\right)^{1/2} \\
\implies \Gamma^2 (3/2) &= \prod_{j=1}^\infty \frac{4j^2 + 4j}{(2j+1)^2} \\
&= \prod_{j=1}^\infty \frac{2j(2j+2)}{(2j+1)^2} \\
&= \frac{1}{2} \left[ 2 \cd \left(\frac{2\cd4}{3^2}\right) \left(\frac{4\cd6}{5^2}\right) \cdots \right] \\
&= \frac{1}{2} \left[ \left(\frac{2\cd2}{1\cd3}\right) \left(\frac{4\cd4}{3\cd5}\right) \left(\frac{6\cd6}{5\cd7}\right) \cdots \right] \\
&= \frac{1}{2} \prod_{j=1}^\infty \frac{(2j)^2}{(2j-1)(2j+1)}.
\end{align*}
This final product that we have is the famous \textit{Wallis Product}, which comes out to $\frac{\pi}{2}$ \cite{WallisProduct}.
Therefore,
$$\Gamma^2 (3/2) = \frac{1}{2} \frac{\pi}{2}$$
\begin{equation}\label{gammaGaussian}
\boxed{\Gamma (3/2) = \frac{\sqrt{\pi}}{2}}.
\end{equation}
It is remarkable that $\pi$ is in this result. 
Where there is $\pi$, there are circles, and where there are circles, there's most likely also quadrature, i.e. integration.
Additionally, those familiar with the Error function will recognize this result as integrating the Gaussian over $\R_+$.
These two facts hint at a possible integral representation for $\Gamma(x)$.

One promising appearance of the factorial in elementary calculus is the power rule.
We know that
$$\frac{d^n}{dx^n} x^n = n!.$$
To use this to construct the desired integral, we would like to use $n$-fold differentiation under the integral sign.
Because of the previous connection to the Gaussian integral, we assume our bounds should be over $\R_+$. 
So far, we have that
$$\Gamma(n+1) \stackrel{?}{=} \int_{0}^\infty x^n \ dx.$$
The immediate issue is that this integral does not converge.
To solve this, we introduce a damping factor $e^{-x}$.
Because we want to generalize this integral to beyond the natural numbers, we will rename $n \mapsto x$ and $x \mapsto t$.

\begin{equation}\label{intForm}
\boxed{\Gamma(x) = \int_{0}^\infty t^{x-1} e^{-t} \ dt}
\end{equation}

Additionally, let's perform a sanity check by using \eqref{intForm} to calculate $\Gamma(3/2)$, and to ensure there is no normalization constant we are missing

\begin{prop}
By \eqref{intForm}, $\Gamma(3/2) = \frac{\sqrt{\pi}}{2}$.
\end{prop}

\begin{proof}
$$\Gamma(3/2) = \int_{0}^\infty \sqrt{t} e^{-t} \ dt.$$
Now we substitute $x = \sqrt{t}$, which implies that $dt = 2x dx$.
Our limits of integration are fixed.
$$\Gamma(3/2) = 2\int_{0}^\infty x^2 e^{-x^2} \ dx.$$
Now let
$$I(\alpha) = \int_{0}^\infty x^2 e^{-\alpha x^2} \ dx.$$
Notice that $2I(1) = \Gamma(3/2)$.
We now differentiate under the integral sign carelessly (the careful mathematician may justify these steps), and find our answer.
$$I(\alpha) = -\int_0^\infty \frac{\partial}{\partial \alpha} e^{-\alpha x^2} \ dx$$
$$I(\alpha) = -\frac{d}{d\alpha} \int_{0}^\infty e^{-\alpha x^2} \ dx.$$
The final integral is a modified Gaussian, which evaluates to $\frac{1}{2}\sqrt{\frac{\pi}{\alpha}}$ \cite{Gaussian}.
$$I(\alpha) = -\frac{\sqrt{\pi}}{2}\frac{d}{d\alpha} \frac{1}{\sqrt{\alpha}}$$
$$I(\alpha) = \frac{\sqrt{\pi}}{4\alpha^{3/2}}.$$
Therefore,
$$\Gamma(3/2) = \frac{\sqrt{\pi}}{2},$$
so \eqref{intForm} agrees with \eqref{EulerForm} in this case.
\end{proof}

Our construction so far has been loose, so we will now ensure that this expression, which we are tentatively calling the $\Gamma$ function, actually has the desired properties.

\begin{lem}
\eqref{intForm} satisfies $\Gamma (n+1) = n!$ for any natural number $n$.
\end{lem}

\begin{proof}
We use our iterated differentiation and differentiation under the integral sign ideas by parameterizing $\Gamma(n,\alpha)$.
Consider the function
$$\Gamma(n+1,\alpha) = \int_0^\infty t^{n} e^{-\alpha t} \ dt$$
for $\alpha > 0$.
Notice that $\Gamma(n+1,1) = \Gamma(n+1)$
We now differentiate under the integral sign with respect to alpha $n$ times, taking care of the resultant sign.
$$\Gamma(n+1, \alpha) = (-1)^n \int_{0}^\infty \frac{\partial^n}{\partial \alpha^n} e^{-\alpha t} \ dt.$$
Now with some trickery (which the careful mathematician can justify), we exchange the order of integration and differentiation.
$$\Gamma(n+1, \alpha) = (-1)^n \frac{d^n}{d\alpha^n} \int_{0}^\infty e^{-\alpha t} \ dt.$$
Now, we are free to compute.
$$\Gamma(n+1, \alpha) = (-1)^n \frac{d^n}{d\alpha^n} \left[\frac{-e^{-\alpha t}}{\alpha}\right]_{0}^\infty$$
$$\Gamma(n+1, \alpha) = (-1)^n \frac{d^n}{d\alpha^n} \left( \frac{1}{\alpha} \right)$$
$$\Gamma(n+1, \alpha) = (-1)^n (-1)^n \frac{n!}{\alpha^{n+1}}$$
$$\Gamma(n+1, \alpha) = \frac{n!}{\alpha^{n+1}}.$$
Recovering our original $\Gamma(x)$ by setting $\alpha = 1$, we find that 
$$\Gamma(n+1) = n!,$$
as required.
\end{proof}

\begin{lem}
\eqref{intForm} satisfies $\Gamma(x+1) = x \Gamma(x)$ for all $x \in \R - \Z_{\leq 0}$.
\end{lem}

\begin{proof}
We know that
$$\Gamma(x+1) = \int_0^\infty t^x e^{-t} \ dt.$$
Now we use integration by parts as follows.
\[
\begin{NiceArray}{c @{\hspace*{1.0cm}} c}[create-medium-nodes]
  \toprule
     D & I \\
  \cmidrule{1-2}
    +t^x & e^{-t} \\
    -xt^{x-1}  & -e^{-t} \\      
  \bottomrule
\CodeAfter
  \begin{tikzpicture} [->, name suffix = -medium]
  \draw [black] (2-1) -- node [above] {} (3-2) ; 
  \draw [black] (3-2) -- node [below] {} (3-1) ;
  \end{tikzpicture}
\end{NiceArray}
\]

Therefore, we find that
$$\Gamma(x+1) = -[t^x e^{-t}]_0^\infty + x \int_{0}^\infty t^{x-1} e^{-t} \ dt$$
$$\Gamma(x+1) = x \Gamma(x),$$
as required.
\end{proof}

Now all we have to prove is logarithmic convexity of \eqref{intForm}.

\begin{lem}
\eqref{intForm} is logarithmically convex on $(0,\infty)$.
\end{lem}

\begin{proof}
From the definition of convexity, all we have to show is that the function
$$\phi_t(x) = t^{x} e^{-t}$$
is logarithmically convex for a given $t$.
Take $a,b \in (0,\infty)$ and $\gamma \in [0,1]$.
Then, we want to show that 
$$\phi_t (\gamma a + (1-\gamma)b) \leq \gamma\phi_t(a) + (1-\gamma)\phi_t(b)$$
$$\iff t^{\gamma a + (1-\gamma)b} e^{-t} \leq \gamma t^{a} e^{-t} + (1-\gamma) t^b e^{-t}$$
$$\iff t^{\gamma a} t^{(1-\gamma)b} \leq \gamma t^a + (1-\gamma) t^b.$$
This is true if and only if the exponential function is convex, which one can prove by taking the second derivative and showing that it is greater than zero.
Therefore, \eqref{intForm} describing $\Gamma(x)$ is convex.
\end{proof}

\begin{thm}
The $\Gamma$ function can be written as 
$$\Gamma(x) = \int_0^\infty t^{x-1} e^{-t} \ dt.$$
\end{thm}

\begin{proof}
We know that $\Gamma(x)$ given by \eqref{intForm} satisfies all the conditions of definition \eqref{BohrMollerupDef} by the previous three lemmas. 
We also know by the Bohr-Mollerup theorem that definition \eqref{BohrMollerupDef} specifies a unique function.
Therefore,
$$\Gamma(x) = \int_0^\infty t^{x-1} e^{-t} \ dx = \lim_{N \rightarrow \infty} \frac{N! N^x}{\prod_{j=0}^N (x+j)}.$$
\end{proof}

Now we may see the importance of the $\Gamma$ function, as we can use these two representations to solve entire classes of summations, products, and integrals.
We will demonstrate this power in the examples section, after we have developed some more tools.

\newpage
\section{Weierstrass and the Reflection Formula}

    When computing with the $\Gamma$ function, one often comes across expressions of the form $\Gamma(x)\Gamma(1-x)$.
One would expect that this is as simple as it can be -- that there is no closed form for such an expression.
That person would be wrong.
Remarkably, this kind of expression has a closed form, using only trigonometric functions!
The goal of this section is to derive this remarkable result, and to explore Weierstrass's form, given in \eqref{WeierstrassFactorial}.

Recall our original definition that 
$$\Gamma(x) = \lim_{N\rightarrow \infty} \frac{N! N^x}{\prod_{j=0}^N (x + j)}.$$
We will manipulate this to recover \eqref{WeierstrassFactorial}.
First, we pull the $j=0$ term, $x$, out of the product, and take reciprocal.
$$\frac{1}{\Gamma(x)} = \lim_{N \rightarrow \infty} x N^{-x} \prod_{j=0}^N \left( \frac{j+x}{j} \right)$$
$$\frac{1}{\Gamma(x)} = \lim_{N \rightarrow \infty} x e^{-x\ln N} \prod_{j=1}^N \left( 1 + \frac{x}{j} \right)$$
Multiplying by a convenient factor of $e^{-x/j} e^{x/j}$ for each $j$ in the product, we find a nice formula in terms of $x/j$.  
$$\frac{1}{\Gamma(x)} = \lim_{N \rightarrow \infty} xe^{-x\ln N} \prod_{j=1}^N \left( 1 + \frac{x}{j} \right) e^{-x/j} e^{x/j}$$
$$\frac{1}{\Gamma(x)} = \lim_{N \rightarrow \infty} xe^{x\sum_{j=1}^N \frac{1}{j} -x\ln N} \prod_{j=1}^N \left( 1 + \frac{x}{j} \right) e^{-x/j} $$
$$\frac{1}{\Gamma(x)} = \lim_{N \rightarrow \infty} xe^{x\left(\sum_{j=1}^N \frac{1}{j} -\ln N\right)} \prod_{j=1}^N \left( 1 + \frac{x}{j} \right) e^{-x/j}.$$

\begin{defn}
The \textit{Euler-Mascheroni constant} $\gamma$ is the limit of the difference of the sum of the first $N$ harmonic terms with the logarithm of $N$.
$$\gamma \equiv \lim_{N \rightarrow \infty} \sum_{j=1}^N \frac{1}{j} - \ln N.$$
\end{defn}
This constant is mysterious, as no one has shown if it is irrational, let alone transcendental \cite{EulerMascheroni}.  

With this new constant in hand, we can see that
\begin{equation}\label{WeierstrassForm}
\boxed{\frac{1}{\Gamma(x)} = xe^{\gamma x} \prod_{j=1}^\infty \left(1 + \frac{x}{j}\right)e^{-x/j}}.
\end{equation}
\eqref{WeierstrassForm} is called the \textit{Weierstrass Form} of the $\Gamma$ function. 
In this form, it is clear that the $\Gamma$ function has poles only on the negative integers.
This form is especially nice because it writes $\Gamma(x)$ using its zeros, similar to how one can factor a polynomial over $\mathbb{C}$.

In fact, there is a similar result for $\sin(x)$, proven by Euler \cite{SinProduct}, that
$$\sin (x) = x \prod_{j=1}^\infty \left(1 - \frac{x^2}{\pi^2 j^2}\right).$$
Sending $x \mapsto \pi x$, we can see that
\begin{equation}\label{SinProductEq}
\sin (\pi x) = \pi x \prod_{j=1}^\infty \left(1 - \frac{x^2}{j^2}\right).
\end{equation}
Using these results, we are now able to prove the Reflection Formula.

\begin{thm}
The Reflection Formula is given by
\begin{equation}\label{ReflectionFormula}
\boxed{\Gamma(x)\Gamma(1-x) = \frac{\pi}{\sin (\pi x)}}.
\end{equation}
\end{thm}

\begin{proof}
We start by using \eqref{WeierstrassForm} to compute the reciprocal of what we want.
$$\frac{1}{\Gamma(x)\Gamma(1-x)} = \frac{1}{-x\Gamma(x)\Gamma(-x)}$$
$$\frac{1}{\Gamma(x)\Gamma(1-x)} = -\frac{1}{x} \left[ xe^{\gamma x} \prod_{j=1}^\infty \left(1 + \frac{x}{j}\right) e^{-x/j} \right] \left[ -x e^{-\gamma x} \prod_{k=1}^\infty \left(1 - \frac{x}{k}\right) e^{x/k} \right].$$
$$\frac{1}{\Gamma(x)\Gamma(1-x)} = x\prod_{j=1}^\infty \left(1 - \frac{x^2}{j^2}\right)$$
We see that this product is \eqref{SinProductEq}, up to a factor of $\pi$.
$$\frac{1}{\Gamma(x)\Gamma(1-x)} = \frac{\sin (\pi x)}{\pi}$$
$$\Gamma(x) \Gamma(1-x) = \frac{\pi}{\sin (\pi x)}$$
\end{proof}

\newpage
\section{The Polygamma Functions}

    Recall our function $L(x) = \ln x! = \ln \Gamma(x+1)$ from \eqref{LFunction}.
Although it was pivotal for our derivation of $\Gamma(x)$, we have so far neglected to explore it.
We are motivated by the fact that $L(x)$ naturally converts the product present in the expression for $\Gamma(x)$ given in \eqref{EulerForm} into a series.
Since series are some of the most troublesome characters in the solutions for many problems, any tools that $L(x)$ may provide which soothe our computational woes are deeply welcome.

Our first course of business is to take care of Euler's domain shift, while keeping the spirit of $L(x)$.
We will define the function
\begin{equation}\label{digammaAntiderivative}
\Psi^{(-1)} (x) \equiv \ln \Gamma(x).
\end{equation}
This perplexing choice of notation for $\Psi^{(-1)}$ will be justified later in this section.
For now, I ask you to simply play along.

Let's see how many properties of $\Psi^{(-1)}$ we can find before diving into any deep computations.
Firstly, our definition immediately implies that
$$\Psi^{(-1)}(x) = L(x-1).$$
From the recursion relation of $\Gamma(x)$, we know that
$$\Gamma(x+1) = x\Gamma(x),$$
so applying our definition for $\Psi^{(-1)}$ gives
\begin{equation}\label{digammaAntiderivativeRecursionFormula}
\Psi^{(-1)} (x+1) = \Psi^{(-1)} (x) + \ln x.
\end{equation}
This gives us an important recursion relation for $\Psi^{(-1)}$.
Finally, from the definition of $\Gamma(x)$, we know that it is log-convex, so $\Psi^{(-1)}$ must be a convex function.

Now let's warm up with some light computations.
Recall the remarkable Reflection Formula for $\Gamma(x)$ from \eqref{ReflectionFormula}, which says that
$$\Gamma(x)\Gamma(1-x) = \frac{\pi}{\sin (\pi x)}.$$
Taking the logarithm on both sides, we see the useful reflection formula
\begin{equation}\label{digammaAntiderivativeReflectionFormula}
\Psi^{(-1)}(x) + \Psi^{(-1)}(1-x) = \ln \pi + \ln \csc (\pi x).
\end{equation}

Recall \eqref{WeierstrassForm}, the Weierstrass form for $\Gamma(x)$, given by
$$\frac{1}{\Gamma(x)} = xe^{\gamma x} \prod_{j=1}^\infty \left( 1 + \frac{x}{j}\right) e^{-x/j}.$$
Applying the definition of $\Psi^{(-1)}(x)$,
\begin{equation}\label{digammaAntiderivativeWeierstrass}
-\Psi^{(-1)} (x) = \ln x + \gamma x + \sum_{j=1}^\infty \ln \left(1 + \frac{x}{j}\right) - \frac{x}{j}.
\end{equation}
We now expand the definition of $\gamma$, given by (4.1).
$$-\Psi^{(-1)} (x) = \ln x + \lim_{N \rightarrow \infty} -x\ln N + \sum_{j=1}^N \left(\frac{x}{j} - \frac{x}{j} + \ln \left( 1 + \frac{x}{j} \right) \right)$$
\begin{equation}\label{digammaAntiderivativeLimit}
\Psi^{(-1)} (x) = -\ln x + \lim_{N \rightarrow \infty} x\ln N - \sum_{j=1}^N \ln \left(1 + \frac{x}{j}\right)
\end{equation}
This is one expression for $\Psi^{(-1)}(x)$ which we will use in the future.

We can check this directly by recalling \eqref{EulerForm}, which gives
$$\Gamma(x) = \lim_{N \rightarrow \infty} \frac{N! N^x}{\prod_{j=0}^N (x+j)}.$$
Using the same procedure as above, we will compute $\Psi^{(-1)}(x)$.
$$\Psi^{(-1)}(x) = \lim_{N \rightarrow \infty} \ln N! + x\ln N - \sum_{j=0}^N \ln (x + j).$$
Expanding the definition of $N!$, we can simplify this expression.
$$\Psi^{(-1)}(x) = \lim_{N \rightarrow \infty} x\ln N - \ln x + \sum_{j=1}^N \ln \left( \frac{j}{x+j} \right)$$
$$\Psi^{(-1)} (x) = -\ln x + \lim_{N \rightarrow \infty} x\ln N - \sum_{j=1}^N \ln \left(1 + \frac{x}{j}\right).$$
This agrees with \eqref{digammaAntiderivativeLimit}, as expected.

So far, $\Psi^{(-1)}(x)$ has not given us anything which we could not have gotten from $\Gamma(x)$.
This is not unexpected, however, because it is simply the logarithm.
To extract new information from $\Psi^{(-1)}$, we can take the derivative.

\begin{defn}
The Digamma Function is given by
\begin{equation}\label{digammaDefinition}
\Psi(x) \equiv \frac{d \Psi^{(-1)}}{dx} = \frac{d \ln \Gamma}{dx}.
\end{equation}
\end{defn}

Now the notation $\Psi^{(-1)}(x)$ takes on the meaning of the anti-derivative of the Digamma function, justifying our previous notation.  

One may reasonably ask why we consider the \textit{logarithmic} derivative of $\Gamma(x)$, and not the regular derivative. 
For one, our expression for $\Gamma(x)$, both in the Eulerian product \eqref{EulerForm} and the Weierstrass form \eqref{WeierstrassForm}, rely on infinite products.
In general, differentiating an infinite product is messy, and does not produce many useful results in this case.
One can, of course, use the integral form \eqref{intForm} to find a simple expression for the derivative of $\Gamma(x)$, which is occasionally useful.
Nevertheless, the derivative is a linear operator, so we want to work with sums, not products.

We will now find a computable expression for \eqref{digammaDefinition}.
Starting with \eqref{digammaAntiderivativeWeierstrass}, we find that 
$$\Psi(x) = -\frac{1}{x} - \gamma + \sum_{j=1}^\infty \frac{1}{j} - \frac{\frac{1}{j}}{1 + \frac{x}{j}}$$
$$\Psi(x) = -\frac{1}{x} - \gamma + \sum_{j=1}^\infty \frac{1}{j} - \frac{1}{x + j}$$
$$\Psi(x) = -\gamma + \sum_{j=1}^\infty \frac{1}{j} - \frac{1}{x + j -1}$$
\begin{equation}\label{digammaEquation}
\boxed{\Psi(x) = -\gamma + \sum_{j=1}^\infty \frac{x - 1}{j (x + j -1)}}.
\end{equation}

Using \eqref{digammaAntiderivativeLimit} instead, we find the following.
$$\Psi(x) = -\frac{1}{x} + \lim_{N \rightarrow \infty} \ln N - \sum_{j=1}^N \frac{\frac{1}{j}}{1 + \frac{x}{j}}$$
$$\Psi(x) = -\frac{1}{x} + \lim_{N \rightarrow \infty} \ln N - \sum_{j=1}^N \frac{1}{x + j}$$
\begin{equation}\label{digammaEquationLimit}
\boxed{\Psi(x) = \lim_{N \rightarrow \infty} \ln N - \sum_{j=0}^N \frac{1}{x + j}}.
\end{equation}

Using \eqref{digammaEquation} and \eqref{digammaEquationLimit}, we can find a closed-form for many classes of sums.
We will go into more depth into this class of problems in the examples section, but for now, we can solve some especially simple cases.

First consider when $x = 1$.
From \eqref{digammaEquation}, we see that $\Psi(1) = -\gamma$.
This can also be seen from \eqref{digammaEquationLimit}, where $x = 1$ corresponds to the negative of the definition of $\gamma$.

Now consider when $x = 2$. 
From \eqref{digammaEquation}, we see that
$$\Psi(2) = -\gamma + \sum_{j=1}^\infty \frac{1}{j^2 + j}$$
$$\Psi(2) = -\gamma + \sum_{j=1}^\infty \frac{1}{j} - \frac{1}{j+1}$$
$$\Psi(2) = 1 - \gamma.$$


Now we will explore the topic of recurrence and reflection formulas for $\Psi(x)$, which have brought us great glory with $\Gamma(x)$.
From \eqref{digammaAntiderivativeRecursionFormula}, we can see that
\begin{equation}\label{digammaRecursionFormula}
\boxed{\Psi(x+1) = \Psi(x) + \frac{1}{x}},
\end{equation}
giving a simple recursion relation.
From \eqref{digammaAntiderivativeReflectionFormula}, we can see that
$$\Psi(x) - \Psi(1-x) = -\frac{d}{dx} \ln \sin (\pi x)$$
$$\Psi(x) = \Psi(1-x) + -\frac{\pi \cos (\pi x)}{\sin (\pi x)}$$
\begin{equation}\label{digammaReflectionFormula}
\boxed{\Psi(x) = \Psi(1-x) + \pi \cot (\pi x)}. 
\end{equation}

Combining the recursion formula for $\Psi(x)$, \eqref{digammaRecursionFormula}, with our observation that $\Psi(1) = -\gamma,$ we see that for $n \in \N$,
$$\Psi(n) = \sum_{j=1}^n \frac{1}{n} - \gamma,$$
showing how the Harmonic Series is connected to the Digamma function.
The appearance of $\gamma$, the definition of which required the Harmonic Series, in our computations hinted at a possible connection between our area of study and that of the Harmonic Series.
This result cements that connection, as $\Psi(x)$ is to the Harmonic Series what $\Gamma(x)$ is to the factorial function. 

Since the study of $\Psi(x)$ has given us some powerful tools, it is natural to see what other results we can squeeze out by taking further derivatives.
This motivated the definition of the \textit{Polygamma Functions}.

\begin{defn}
The Polygamma Function of order $n$ is given by 
\begin{equation}\label{polygammaDefinition}
\Psi_n (x) \equiv \frac{d^n}{dx^n} \Psi(x) = \frac{d^{n+1}}{dx^{n+1}} (\ln \Gamma(x)),
\end{equation}
where $\Psi_0 (x) \equiv \Psi(x)$.
\end{defn}

It is not so tedious to derive a closed form for \eqref{polygammaDefinition} if we start from \eqref{digammaEquationLimit}.
If we start with \eqref{digammaEquation} instead, we need to handle the quotient of products in the infinite series, creating a tractable but ultimately unnecessary mess.
$$\Psi_n(x) = \frac{d^n}{dx^n} \Psi(x)$$
$$\Psi_n(x) = \frac{d^n}{dx^n} \left[ \lim_{N \rightarrow \infty} \ln N - \sum_{j=0}^N \frac{1}{x+j} \right]$$
$$\Psi_n(x) = -\frac{d^n}{dx^n} \sum_{j=0}^\infty \frac{1}{x+j} \hspace{10px} (\text{Assuming $n \geq 1$})$$
\begin{equation}\label{polygammaEquation}
\boxed{\Psi_n(x) = \sum_{j=0}^\infty \frac{(-1)^{n+1} n!}{(x+j)^{n+1}}} \hspace{10px} \text{for $n \geq 1$}.
\end{equation}

We will finish this section by using \eqref{digammaRecursionFormula} to find a recursion formula for $\Psi_n (x)$.
We can see that
\begin{equation}\label{polygammaRecursionFormula}
\boxed{\Psi_n (x+1) = \Psi_n (x) + \frac{(-1)^{n} n!}{x^{n+1}}}.
\end{equation}
Notice the striking similarity to \eqref{polygammaEquation}, due to the artifact of repeated differentiation of a power, which we had originally used to find the integral form for $\Gamma(x)$.

\newpage
\section{The Beta Function}

    Recall the integral form of $\Gamma(x)$ given in \eqref{intForm} by
$$\Gamma(x) = \int_{0}^\infty t^{x-1} e^{-t} \ dt.$$
This lets us solve a class of problems which is otherwise intractable. 
One such class of integrals is given by the Beta function, which we will derive now.

Consider the product of $\Gamma$ functions
$$\Gamma(x)\Gamma(y) = \int_{0}^\infty \mu^{x-1} e^{-\mu} \ d\mu \int_{0}^\infty \nu^{y-1} e^{-\nu} \ d\nu.$$
Now make the substitution $\mu = u^2, \nu = v^2,$ which means that $d\mu = 2 u du$ and $d\nu = 2 vdv.$
Recognizing that our bounds are unchanged, we find the following.
$$\Gamma(x)\Gamma(y) = 4\int_{0}^\infty u^{2x-1} e^{-u^2} \ du \int_{0}^\infty v^{2y-1} e^{-v^2} \ dv$$
Carelessly, we can combine these integrals.
$$\Gamma(x)\Gamma(y) = 4\int_{0}^\infty \int_{0}^\infty u^{2x-1} v^{2y-1} e^{-(u^2 + v^2)} \ du \ dv.$$
To get rid of the factor of four, we can exploit the even property of the exponential.
By taking the absolute value of $u$ and $v$, while simultaneously expanding the bounds to all of $\R$, we may absorb the four into our integrals directly.
$$\Gamma(x)\Gamma(y) = \int_{-\infty}^\infty \int_{-\infty}^\infty |u|^{2x-1} |v|^{2y-1} e^{-(u^2 + v^2)} \ du \ dv$$
The $u^2 + v^2$ in the resultant integral motivates us to transform into polar coordinates.
This means that $u = r \cos \theta, v = r \sin \theta$, and $du dv = r dr d\theta$.
Taking care of our bounds, we find the following integral.
$$\Gamma(x)\Gamma(y) =  \int_{0}^\infty \int_{0}^{2\pi} r |r\cos \theta|^{2x-1} |r\sin \theta|^{2y-1} e^{-r^2} \ dr \ d\theta$$
We now pull out all of the $r$ dependence. 
$$\Gamma(x)\Gamma(y) = \int_{0}^\infty r^{2x + 2y - 1} e^{-r^2} \ dr \int_{0}^{2\pi} |\cos \theta|^{2x-1} |\sin \theta|^{2y-1} \ d\theta$$
In the radial integral, making the substitution $t = r^2$, we see the leading integral to be nothing but $\Gamma(x + y)/2$.
This gives a promising result.
$$\Gamma(x)\Gamma(y) = \frac{\Gamma(x+y)}{2} \int_{0}^{2\pi} |\cos \theta|^{2x-1} |\sin \theta|^{2y-1} \ d\theta$$
The final trick to clean this up comes from noticing that integrating over the whole unit circle of the absolute values of sine and cosine is equivalent to integrating over the first quadrant four times.
Therefore,
$$\Gamma(x)\Gamma(y) = 2\Gamma(x+y) \int_{0}^{\frac{\pi}{2}} \cos^{2x-1} (\theta) \sin^{2y-1} (\theta) \ d\theta.$$

The integral present in this formula is not trivial to compute for general $x$ and $y$.
Then, we are only one definition away from solving this whole class of integrals.

\begin{defn}
The Beta function is given by
\begin{equation}\label{betaEquation}
\boxed{B(x,y) \equiv 2 \int_{0}^{\frac{\pi}{2}} \cos^{2x-1} \theta \sin^{2x-1} \theta \ d\theta}
\end{equation}
\end{defn}

The fact that $B$ is defined in terms of two variables makes it even more useful, as we have another degree of freedom to parameterize integrals.

The motivation for this definition is clear from the previous computation -- it's connection to $\Gamma(x)$.
We find the most important formula for $B$ as below.
\begin{equation}\label{betaGamma}
\boxed{B(x,y) = \frac{\Gamma(x)\Gamma(y)}{\Gamma(x+y)}}
\end{equation}

We leave finding the product form, recursion formula, and reflection formula for $B$ as an exercise for the reader.
These computations are very similar to those of the Polygamma functions.

Finally, we convert \eqref{betaEquation} into a different, more applicable form.
We know that 
$$B(x,y) = 2 \int_{0}^{\frac{\pi}{2}} \cos^{2x-1} \theta \sin ^{2y-1} \theta \ d\theta.$$
We first use the Pythagorean identity $\sin^2 \theta = 1 - \cos^2 \theta$.
$$B(x,y) = 2 \int_{0}^{\frac{\pi}{2}} (\cos^{2})^{x-1/2} \theta (1 - \cos^2 \theta)^{y-1/2} \ d\theta$$
Now we let $u = \cos^2 \theta$, so $\theta = \arccos \sqrt{u} \implies d\theta = \frac{-du}{2\sqrt{u}\sqrt{1 - u}}.$
As for the bounds, as $\theta \rightarrow 0$, $u \rightarrow 1$, and as $\theta \rightarrow \frac{\pi}{2}$, $u \rightarrow 0$.
This gives us the following.
$$B(x,y) = -2 \int_{1}^0 u^{x - \frac{1}{2}} (1 - u)^{y - \frac{1}{2}} \frac{1}{2\sqrt{u}\sqrt{1-u}} \ du$$
We use the negative sign to flip the bounds, and find the final form we desire.

\begin{equation}\label{betaEquationPolynomial}
\boxed{B(x,y) = \int_0^1 u^{x-1} (1-u)^{y-1} \ du}
\end{equation}



\newpage
\section{Example Computations Using the Gamma Function}

    This section will be a collection of results obtainable using the tools we have developed so far.
We start by considering the Bose-Einstein integrals, which are inspired by the quantum statistics of bosons.

\begin{center}
\textit{\textbf{Problem One: The Bose-Einstein Integral}}
\end{center}

Let the parameterized Bose-Einstein integral be
$$I_{BE} (s) = \int_{0}^\infty \frac{x^s}{e^{x-t} - 1} \ dx$$
for a constant $t < 0$. 
Now expand by $e^{-x}$ on top and bottom.
$$I_{BE} (s) = \int_{0}^\infty \frac{e^{-x} x^s}{e^{-t} - e^{-x}} \ dx$$
The structure inside the integral reminds us of a geometric series, which we realize by factoring $e^{-t}$.
$$I_{BE} (s) = e^{t} \int_0^\infty \frac{e^{-x} x^s}{1 - e^{t-x}} \ dx$$
Now recall that the geometric series is given by 
$$\sum_{j=0}^\infty x^j = \frac{1}{1 - x} \text{ for } |x| < 1.$$
We use this to transform the integral.
$$I_{BE} (s) = e^{t} \int_{0}^\infty e^{-x} x^s \sum_{j=0}^\infty (e^{t-x})^j \ dx$$
Since we have a damping exponential, we are free to reorder the sum and integral as we please.
$$I_{BE} (s) = \sum_{j=0}^\infty e^{t(j+1)} \int_{0}^\infty x^s e^{-x(j+1)} \ dx$$
Noticing the similarity of the integral with the $\Gamma$ function, we make the substitution $u = x (j+1)$.
Using the integral form, \eqref{intForm}, we find the following.
$$I_{BE} (s) = \sum_{j=0}^\infty \frac{e^{t(j+1)}}{(j+1)^{s+1}} \Gamma(s+1)$$
Sending $j \mapsto j+1$ and using the recursion formula for $\Gamma$, we find our answer.
$$\boxed{I_{BE} (s) = s\Gamma(s) \sum_{j=1}^\infty \frac{e^{tj}}{j^{s+1}}}$$

\begin{center}
\textit{\textbf{Problem Two: A Factorial Sum}}
\end{center}

Consider the sum given by
$$S_{!} = \sum_{j=0}^\infty \frac{j!}{(2j+1)!}.$$
We can use the $\Gamma$ function to rewrite this sum.
$$S_{!} = \sum_{j=0}^\infty \frac{\Gamma(j+1)}{\Gamma(2j+2)}$$
$$S_{!} = \sum_{j=0}^\infty \frac{\Gamma(j+1)}{\Gamma((j+1) + (j+1))}$$
$$S_{!} = \sum_{j=0}^\infty \frac{\Gamma(j+1) \Gamma(j+1)}{\Gamma((j+1)+(j+1)) \Gamma(j+1)}.$$
The structure inside the sum reminds us of the $B$ function formula \eqref{betaGamma}.
We use this to reduce the sum.
$$S_{!} = \sum_{j=0}^\infty \frac{B(j+1,j+1)}{\Gamma(j+1)}$$
Now we invoke the polynomial formula for the $B$ function, \eqref{betaEquationPolynomial}.
$$S_{!} = \sum_{j=0}^\infty \frac{1}{\Gamma(j+1)} \int_0^1 x^{j} (1-x)^j \ dx$$
$$S_{!} = \sum_{j=0}^\infty \frac{1}{\Gamma(j+1)} \int_0^1 (x(1-x))^j \ dx$$
Carelessly, we swap the summation and integration.
$$S_{!} = \int_0^1 \sum_{j=0}^\infty \frac{(x(1-x))^j}{\Gamma(j+1)} \ dx$$
$$S_{!} = \int_0^1 \sum_{j=0}^\infty \frac{(x(1-x))^j}{j!} \ dx$$
Now recall that the Taylor expansion of the exponential is given by 
$$e^x = \sum_{j=0}^\infty \frac{x^j}{j!}.$$
We can use this to reduce our sum.
$$S_{!} = \int_0^1 e^{x(1-x)} \ dx$$
$$S_{!} = \int_0^1 e^{-(x-\frac{1}{2})^2 + \frac{1}{4}} \ dx$$
$$S_{!} = e^{\frac{1}{4}} \int_0^1 e^{-(x-\frac{1}{2})^2} \ dx$$
Now let $t = x-\frac{1}{2}$.
$$S_{!} = e^{\frac{1}{4}} \int_{-\frac{1}{2}}^{\frac{1}{2}} e^{-t^2} \ dt$$
We now leverage the even property of the integrand to change the bounds.
$$S_{!} = 2 \sqrt{\sqrt{e}} \int_{0}^{\frac{1}{2}} e^{-t^2} \ dt$$
We can leave our answer here if we wish, as this integral can be numerically computed much quicker than our original sum.

Those familiar with the Error function, however, will notice this integral as a modified version.
Using the formula
$$\text{Erf}(x) = \frac{2}{\sqrt{\pi}} \int_0^x e^{-t^2} \ dt,$$
we can say that
$$\boxed{S_{!} = \sqrt{\pi\sqrt{e}} \ \text{Erf} \left(\frac{1}{2}\right)}.$$

\begin{center}
\textit{\textbf{Problem Three: The Bernoulli Integral}}
\end{center}

Consider the integral 
$$I_B = \int_0^1 x^x \ dx$$
We start by rewriting the exponential.
$$I_B = \int_0^1 e^{x\ln x} \ dx$$
Now we invoke the Taylor expansion of the exponential.
$$I_B = \int_0^1 \sum_{j=0}^\infty \frac{x^j \ln^j x}{j!} \ dx$$
Carelessly, we swap and the summation and integration.
$$I_B = \sum_{j=0}^\infty \frac{1}{j!} \int_0^1 x^j \ln^j x \ dx$$
We perform the substitution $t = -\ln x \iff x = e^{-t} \implies dx = -e^{-t} \ dt$.
Suitably changing bounds, we find that 
$$I_B = \sum_{j=0}^\infty \frac{1}{j!} \int_{\infty}^0 e^{-jt} (-1)^j t^j (-e^{-jt}) \ dt$$
$$I_B = \sum_{j=0}^\infty \frac{(-1)^j}{j!} \int_{0}^\infty t^j e^{-(j+1)t} \ dt$$
Using a similar substitution to the one at the end of problem one, we can see that 
$$I_B = \sum_{j=0}^\infty \frac{(-1)^j}{j! (j+1)^{j+1}} \int_0^\infty t^j e^{-t} \ dt$$
This final integral is \eqref{intForm}, so we find a series to evaluate.
$$I_B = \sum_{j=0}^\infty \frac{(-1)^j j!}{j! (j+1)^{j+1}}$$
$$\boxed{I_B = \sum_{j=1}^\infty \frac{(-1)^{j+1}}{j^j}}$$
As an exercise, one may compute the generalized
$$I_B (\alpha) = \int_0^1 x^{x^\alpha} \ dx$$
using this method.

\begin{thebibliography}{6}

    \bibitem{uniquenessPF}
    \href{https://christophborgers.com/bohr-mollerup-theorem}{Proof of the Bohr-Mollerup Theorem} (Christoph B\"{o}rgers)

    \bibitem{uniquenessPF2}
    \href{https://dspace.mit.edu/bitstream/handle/1721.1/78574/18-100c-spring-2006/contents/projects/duong.pdf}{Another Proof of the Boh-Molleup Theorem} (Thu Ngoc Duong)

    \bibitem{WallisProduct}
    \href{https://www.diva-portal.org/smash/get/diva2:375184/FULLTEXT01.pdf}{History and proof of the Wallis Product} (Johan W\"{a}stlund)

    \bibitem{Gaussian}
    \href{https://kconrad.math.uconn.edu/blurbs/analysis/gaussianintegral.pdf}{Five proofs of the Gaussian Integral} (Keith Conrad)

    \bibitem{SinProduct}
    \href{https://www.sciencedirect.com/science/article/pii/0022247X77902347?ref=pdf\_download\&fr=RR-2\&rr=8a870c785b50e21b}{Proof of sin product formula} (W. F. Eberlin)

    \bibitem{EulerMascheroni}
    \href{https://mathworld.wolfram.com/Euler-MascheroniConstant.html}{Properties of $\gamma$} (Wolfram Mathworld)

\end{thebibliography}

\end{document}

