This section will be a collection of results obtainable using the tools we have developed so far.
We start by considering the Bose-Einstein integrals, which are inspired by the quantum statistics of bosons.

\begin{center}
\textit{\textbf{Problem One: The Bose-Einstein Integral}}
\end{center}

Let the parameterized Bose-Einstein integral be
$$I_{BE} (s) = \int_{0}^\infty \frac{x^s}{e^{x-t} - 1} \ dx$$
for a constant $t < 0$. 
Now expand by $e^{-x}$ on top and bottom.
$$I_{BE} (s) = \int_{0}^\infty \frac{e^{-x} x^s}{e^{-t} - e^{-x}} \ dx$$
The structure inside the integral reminds us of a geometric series, which we realize by factoring $e^{-t}$.
$$I_{BE} (s) = e^{t} \int_0^\infty \frac{e^{-x} x^s}{1 - e^{t-x}} \ dx$$
Now recall that the geometric series is given by 
$$\sum_{j=0}^\infty x^j = \frac{1}{1 - x} \text{ for } |x| < 1.$$
We use this to transform the integral.
$$I_{BE} (s) = e^{t} \int_{0}^\infty e^{-x} x^s \sum_{j=0}^\infty (e^{t-x})^j \ dx$$
Since we have a damping exponential, we are free to reorder the sum and integral as we please.
$$I_{BE} (s) = \sum_{j=0}^\infty e^{t(j+1)} \int_{0}^\infty x^s e^{-x(j+1)} \ dx$$
Noticing the similarity of the integral with the $\Gamma$ function, we make the substitution $u = x (j+1)$.
Using the integral form, \eqref{intForm}, we find the following.
$$I_{BE} (s) = \sum_{j=0}^\infty \frac{e^{t(j+1)}}{(j+1)^{s+1}} \Gamma(s+1)$$
Sending $j \mapsto j+1$ and using the recursion formula for $\Gamma$, we find our answer.
$$\boxed{I_{BE} (s) = s\Gamma(s) \sum_{j=1}^\infty \frac{e^{tj}}{j^{s+1}}}$$

\begin{center}
\textit{\textbf{Problem Two: A Factorial Sum}}
\end{center}

Consider the sum given by
$$S_{!} = \sum_{j=0}^\infty \frac{j!}{(2j+1)!}.$$
We can use the $\Gamma$ function to rewrite this sum.
$$S_{!} = \sum_{j=0}^\infty \frac{\Gamma(j+1)}{\Gamma(2j+2)}$$
$$S_{!} = \sum_{j=0}^\infty \frac{\Gamma(j+1)}{\Gamma((j+1) + (j+1))}$$
$$S_{!} = \sum_{j=0}^\infty \frac{\Gamma(j+1) \Gamma(j+1)}{\Gamma((j+1)+(j+1)) \Gamma(j+1)}.$$
The structure inside the sum reminds us of the $B$ function formula \eqref{betaGamma}.
We use this to reduce the sum.
$$S_{!} = \sum_{j=0}^\infty \frac{B(j+1,j+1)}{\Gamma(j+1)}$$
Now we invoke the polynomial formula for the $B$ function, \eqref{betaEquationPolynomial}.
$$S_{!} = \sum_{j=0}^\infty \frac{1}{\Gamma(j+1)} \int_0^1 x^{j} (1-x)^j \ dx$$
$$S_{!} = \sum_{j=0}^\infty \frac{1}{\Gamma(j+1)} \int_0^1 (x(1-x))^j \ dx$$
Carelessly, we swap the summation and integration.
$$S_{!} = \int_0^1 \sum_{j=0}^\infty \frac{(x(1-x))^j}{\Gamma(j+1)} \ dx$$
$$S_{!} = \int_0^1 \sum_{j=0}^\infty \frac{(x(1-x))^j}{j!} \ dx$$
Now recall that the Taylor expansion of the exponential is given by 
$$e^x = \sum_{j=0}^\infty \frac{x^j}{j!}.$$
We can use this to reduce our sum.
$$S_{!} = \int_0^1 e^{x(1-x)} \ dx$$
$$S_{!} = \int_0^1 e^{-(x-\frac{1}{2})^2 + \frac{1}{4}} \ dx$$
$$S_{!} = e^{\frac{1}{4}} \int_0^1 e^{-(x-\frac{1}{2})^2} \ dx$$
Now let $t = x-\frac{1}{2}$.
$$S_{!} = e^{\frac{1}{4}} \int_{-\frac{1}{2}}^{\frac{1}{2}} e^{-t^2} \ dt$$
We now leverage the even property of the integrand to change the bounds.
$$S_{!} = 2 \sqrt{\sqrt{e}} \int_{0}^{\frac{1}{2}} e^{-t^2} \ dt$$
We can leave our answer here if we wish, as this integral can be numerically computed much quicker than our original sum.

Those familiar with the Error function, however, will notice this integral as a modified version.
Using the formula
$$\text{Erf}(x) = \frac{2}{\sqrt{\pi}} \int_0^x e^{-t^2} \ dt,$$
we can say that
$$\boxed{S_{!} = \sqrt{\pi\sqrt{e}} \ \text{Erf} \left(\frac{1}{2}\right)}.$$

\begin{center}
\textit{\textbf{Problem Three: The Bernoulli Integral}}
\end{center}

Consider the integral 
$$I_B = \int_0^1 x^x \ dx$$
We start by rewriting the exponential.
$$I_B = \int_0^1 e^{x\ln x} \ dx$$
Now we invoke the Taylor expansion of the exponential.
$$I_B = \int_0^1 \sum_{j=0}^\infty \frac{x^j \ln^j x}{j!} \ dx$$
Carelessly, we swap and the summation and integration.
$$I_B = \sum_{j=0}^\infty \frac{1}{j!} \int_0^1 x^j \ln^j x \ dx$$
We perform the substitution $t = -\ln x \iff x = e^{-t} \implies dx = -e^{-t} \ dt$.
Suitably changing bounds, we find that 
$$I_B = \sum_{j=0}^\infty \frac{1}{j!} \int_{\infty}^0 e^{-jt} (-1)^j t^j (-e^{-jt}) \ dt$$
$$I_B = \sum_{j=0}^\infty \frac{(-1)^j}{j!} \int_{0}^\infty t^j e^{-(j+1)t} \ dt$$
Using a similar substitution to the one at the end of problem one, we can see that 
$$I_B = \sum_{j=0}^\infty \frac{(-1)^j}{j! (j+1)^{j+1}} \int_0^\infty t^j e^{-t} \ dt$$
This final integral is \eqref{intForm}, so we find a series to evaluate.
$$I_B = \sum_{j=0}^\infty \frac{(-1)^j j!}{j! (j+1)^{j+1}}$$
$$\boxed{I_B = \sum_{j=1}^\infty \frac{(-1)^{j+1}}{j^j}}$$
As an exercise, one may compute the generalized
$$I_B (\alpha) = \int_0^1 x^{x^\alpha} \ dx$$
using this method.