We will start with by using our newfound knowledge to compute $(1/2)! = \Gamma(3/2)$.
First, however, we need to make a slight modification to our formula to recover Euler's historically first expression for $\Gamma(x)$.

\begin{prop}
\begin{equation}\label{historicProductForm}
\Gamma(x+1) = \prod_{j=1}^\infty \left( \frac{\left(1 + \frac{1}{j}\right)^x}{1 + \frac{x}{j}}\ \right).
\end{equation}
\end{prop}

\begin{proof}
First recall that
$$\Gamma(x+1) = \lim_{N \rightarrow \infty} N^x \prod_{j=1}^N \left( \frac{j}{j+x} \right).$$
Dividing through by $j$ inside the product, we find
$$\Gamma(x+1) = \lim_{N \rightarrow \infty} N^x \prod_{j=1}^N \frac{1}{1 + \frac{x}{j}}.$$
All that is left to show now is that, in the limit,
$$N^x \approx \prod_{j=1}^N \left(1 + \frac{1}{j}\right)^x.$$
This is true if
$$\prod_{j=1}^N \left(1 + \frac{1}{j}\right) = N+1,$$
which we will show by induction on $N$.
The base case of $N=1$ is clear.
For the inductive step, take some $k \in \N$ such that 
$$\prod_{j=1}^k \left(1 + \frac{1}{j}\right) = k+1.$$
Then, the product up to $k+1$ can be manipulated as follows.
\begin{align*}
\prod_{j=1}^{k+1} \left(1 + \frac{1}{j}\right) &= \left(1+\frac{1}{k+1}\right)\prod_{j=1}^{k} \left(1 + \frac{1}{j}\right) \\
&= \left(1+\frac{1}{k+1}\right) (k+1) \\
&= (k+1) + 1.
\end{align*}
Therefore, our inductive step is fulfilled, our desired limit is true, and
$$\Gamma(x+1) = \prod_{j=1}^\infty \left( \frac{\left(1 + \frac{1}{j}\right)^{x}}{1 + \frac{x}{j}} \right).$$
\end{proof}

Now, computing $(1/2)!$ is a matter of substituting $x=1/2$ into \eqref{historicProductForm}.
\begin{align*}
\Gamma(3/2) &= \prod_{j=1}^\infty \frac{\left(1 + \frac{1}{j}\right)^{1/2}}{1 + \frac{1}{2j}} \\
&= \left(\prod_{j=1}^\infty \frac{1 + \frac{1}{j}}{\left(1 + \frac{1}{2j}\right)^2}\right)^{1/2} \\
\implies \Gamma^2 (3/2) &= \prod_{j=1}^\infty \frac{4j^2 + 4j}{(2j+1)^2} \\
&= \prod_{j=1}^\infty \frac{2j(2j+2)}{(2j+1)^2} \\
&= \frac{1}{2} \left[ 2 \cd \left(\frac{2\cd4}{3^2}\right) \left(\frac{4\cd6}{5^2}\right) \cdots \right] \\
&= \frac{1}{2} \left[ \left(\frac{2\cd2}{1\cd3}\right) \left(\frac{4\cd4}{3\cd5}\right) \left(\frac{6\cd6}{5\cd7}\right) \cdots \right] \\
&= \frac{1}{2} \prod_{j=1}^\infty \frac{(2j)^2}{(2j-1)(2j+1)}.
\end{align*}
This final product that we have is the famous \textit{Wallis Product}, which comes out to $\frac{\pi}{2}$ \cite{WallisProduct}.
Therefore,
$$\Gamma^2 (3/2) = \frac{1}{2} \frac{\pi}{2}$$
\begin{equation}\label{gammaGaussian}
\boxed{\Gamma (3/2) = \frac{\sqrt{\pi}}{2}}.
\end{equation}
It is remarkable that $\pi$ is in this result. 
Where there is $\pi$, there are circles, and where there are circles, there's most likely also quadrature, i.e. integration.
Additionally, those familiar with the Error function will recognize this result as integrating the Gaussian over $\R_+$.
These two facts hint at a possible integral representation for $\Gamma(x)$.

One promising appearance of the factorial in elementary calculus is the power rule.
We know that
$$\frac{d^n}{dx^n} x^n = n!.$$
To use this to construct the desired integral, we would like to use $n$-fold differentiation under the integral sign.
Because of the previous connection to the Gaussian integral, we assume our bounds should be over $\R_+$. 
So far, we have that
$$\Gamma(n+1) \stackrel{?}{=} \int_{0}^\infty x^n \ dx.$$
The immediate issue is that this integral does not converge.
To solve this, we introduce a damping factor $e^{-x}$.
Because we want to generalize this integral to beyond the natural numbers, we will rename $n \mapsto x$ and $x \mapsto t$.

\begin{equation}\label{intForm}
\boxed{\Gamma(x) = \int_{0}^\infty t^{x-1} e^{-t} \ dt}
\end{equation}

Additionally, let's perform a sanity check by using \eqref{intForm} to calculate $\Gamma(3/2)$, and to ensure there is no normalization constant we are missing

\begin{prop}
By \eqref{intForm}, $\Gamma(3/2) = \frac{\sqrt{\pi}}{2}$.
\end{prop}

\begin{proof}
$$\Gamma(3/2) = \int_{0}^\infty \sqrt{t} e^{-t} \ dt.$$
Now we substitute $x = \sqrt{t}$, which implies that $dt = 2x dx$.
Our limits of integration are fixed.
$$\Gamma(3/2) = 2\int_{0}^\infty x^2 e^{-x^2} \ dx.$$
Now let
$$I(\alpha) = \int_{0}^\infty x^2 e^{-\alpha x^2} \ dx.$$
Notice that $2I(1) = \Gamma(3/2)$.
We now differentiate under the integral sign carelessly (the careful mathematician may justify these steps), and find our answer.
$$I(\alpha) = -\int_0^\infty \frac{\partial}{\partial \alpha} e^{-\alpha x^2} \ dx$$
$$I(\alpha) = -\frac{d}{d\alpha} \int_{0}^\infty e^{-\alpha x^2} \ dx.$$
The final integral is a modified Gaussian, which evaluates to $\frac{1}{2}\sqrt{\frac{\pi}{\alpha}}$ \cite{Gaussian}.
$$I(\alpha) = -\frac{\sqrt{\pi}}{2}\frac{d}{d\alpha} \frac{1}{\sqrt{\alpha}}$$
$$I(\alpha) = \frac{\sqrt{\pi}}{4\alpha^{3/2}}.$$
Therefore,
$$\Gamma(3/2) = \frac{\sqrt{\pi}}{2},$$
so \eqref{intForm} agrees with \eqref{EulerForm} in this case.
\end{proof}

Our construction so far has been loose, so we will now ensure that this expression, which we are tentatively calling the $\Gamma$ function, actually has the desired properties.

\begin{lem}
\eqref{intForm} satisfies $\Gamma (n+1) = n!$ for any natural number $n$.
\end{lem}

\begin{proof}
We use our iterated differentiation and differentiation under the integral sign ideas by parameterizing $\Gamma(n,\alpha)$.
Consider the function
$$\Gamma(n+1,\alpha) = \int_0^\infty t^{n} e^{-\alpha t} \ dt$$
for $\alpha > 0$.
Notice that $\Gamma(n+1,1) = \Gamma(n+1)$
We now differentiate under the integral sign with respect to alpha $n$ times, taking care of the resultant sign.
$$\Gamma(n+1, \alpha) = (-1)^n \int_{0}^\infty \frac{\partial^n}{\partial \alpha^n} e^{-\alpha t} \ dt.$$
Now with some trickery (which the careful mathematician can justify), we exchange the order of integration and differentiation.
$$\Gamma(n+1, \alpha) = (-1)^n \frac{d^n}{d\alpha^n} \int_{0}^\infty e^{-\alpha t} \ dt.$$
Now, we are free to compute.
$$\Gamma(n+1, \alpha) = (-1)^n \frac{d^n}{d\alpha^n} \left[\frac{-e^{-\alpha t}}{\alpha}\right]_{0}^\infty$$
$$\Gamma(n+1, \alpha) = (-1)^n \frac{d^n}{d\alpha^n} \left( \frac{1}{\alpha} \right)$$
$$\Gamma(n+1, \alpha) = (-1)^n (-1)^n \frac{n!}{\alpha^{n+1}}$$
$$\Gamma(n+1, \alpha) = \frac{n!}{\alpha^{n+1}}.$$
Recovering our original $\Gamma(x)$ by setting $\alpha = 1$, we find that 
$$\Gamma(n+1) = n!,$$
as required.
\end{proof}

\begin{lem}
\eqref{intForm} satisfies $\Gamma(x+1) = x \Gamma(x)$ for all $x \in \R - \Z_{\leq 0}$.
\end{lem}

\begin{proof}
We know that
$$\Gamma(x+1) = \int_0^\infty t^x e^{-t} \ dt.$$
Now we use integration by parts as follows.
\[
\begin{NiceArray}{c @{\hspace*{1.0cm}} c}[create-medium-nodes]
  \toprule
     D & I \\
  \cmidrule{1-2}
    +t^x & e^{-t} \\
    -xt^{x-1}  & -e^{-t} \\      
  \bottomrule
\CodeAfter
  \begin{tikzpicture} [->, name suffix = -medium]
  \draw [black] (2-1) -- node [above] {} (3-2) ; 
  \draw [black] (3-2) -- node [below] {} (3-1) ;
  \end{tikzpicture}
\end{NiceArray}
\]

Therefore, we find that
$$\Gamma(x+1) = -[t^x e^{-t}]_0^\infty + x \int_{0}^\infty t^{x-1} e^{-t} \ dt$$
$$\Gamma(x+1) = x \Gamma(x),$$
as required.
\end{proof}

Now all we have to prove is logarithmic convexity of \eqref{intForm}.

\begin{lem}
\eqref{intForm} is logarithmically convex on $(0,\infty)$.
\end{lem}

\begin{proof}
From the definition of convexity, all we have to show is that the function
$$\phi_t(x) = t^{x} e^{-t}$$
is logarithmically convex for a given $t$.
Take $a,b \in (0,\infty)$ and $\gamma \in [0,1]$.
Then, we want to show that 
$$\phi_t (\gamma a + (1-\gamma)b) \leq \gamma\phi_t(a) + (1-\gamma)\phi_t(b)$$
$$\iff t^{\gamma a + (1-\gamma)b} e^{-t} \leq \gamma t^{a} e^{-t} + (1-\gamma) t^b e^{-t}$$
$$\iff t^{\gamma a} t^{(1-\gamma)b} \leq \gamma t^a + (1-\gamma) t^b.$$
This is true if and only if the exponential function is convex, which one can prove by taking the second derivative and showing that it is greater than zero.
Therefore, \eqref{intForm} describing $\Gamma(x)$ is convex.
\end{proof}

\begin{thm}
The $\Gamma$ function can be written as 
$$\Gamma(x) = \int_0^\infty t^{x-1} e^{-t} \ dt.$$
\end{thm}

\begin{proof}
We know that $\Gamma(x)$ given by \eqref{intForm} satisfies all the conditions of definition \eqref{BohrMollerupDef} by the previous three lemmas. 
We also know by the Bohr-Mollerup theorem that definition \eqref{BohrMollerupDef} specifies a unique function.
Therefore,
$$\Gamma(x) = \int_0^\infty t^{x-1} e^{-t} \ dx = \lim_{N \rightarrow \infty} \frac{N! N^x}{\prod_{j=0}^N (x+j)}.$$
\end{proof}

Now we may see the importance of the $\Gamma$ function, as we can use these two representations to solve entire classes of summations, products, and integrals.
We will demonstrate this power in the examples section, after we have developed some more tools.